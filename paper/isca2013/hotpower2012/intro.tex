\section{Introduction}

Although smartphones dominate mobile platforms, encroaching the turf of traditional computing platforms such
as laptop and desktop systems, the primary factor limiting their mobility is the energy efficiency and thus the battery 
life. Among the components in the smartphones, processors have been considered to be one of the top two power-hungry
components, along with the display device~\cite{smartphonepower,intothewild}. Meanwhile, the processors for smartphones have 
evolved into multi-cores, making the exploitation of thread-level parallelism (TLP) critical 
for the performance and energy efficiency~\cite{desktopTLP}. To fully utilize the potential benefit of multi-cores, and 
to improve the energy efficiency, it is critical to understand the behaviors of smartphone applications for their
available TLP.

However, such smartphone applications exhibit quite different behaviors from the traditional benchmark applications
used for servers and desktops. Their executions are highly dependent upon the interactions with users, and the
power consumption behaviors are also affected by user interactions. Although the importance of understanding such
interactive applications has been growing, there  have been a limited number of studies on the behavior of 
smartphone applications~\cite{intothewild, characterinteractive}.

This paper characterizes three different aspects of power behaviors in commercial smartphone platforms and 
their applications. Firstly, this study analyzes the actual power consumption of multi-cores used for
smartphones with varying clock frequencies and CPU utilization. We used two types of micro-benchmarks,
one for stressing CPUs and the other for stressing the memory system. The analysis shows how the 
current multi-cores and their support for dynamic voltage and frequency scaling (DVFS) consume power 
with the two extreme cases of application types.

Secondly, we characterize the parallelism and CPU utilization behaviors of smartphone applications.
Gutierrez et al. showed the characteristics of smartphone applications, focusing their micro-architectural
aspects~\cite{characterinteractive}. This paper focuses on the thread-level parallelism (TLP) and 
CPU utilization, which are directly related to the power management with DVFS and dynamic core plug-ins~\cite{packandcap}.
We use user interaction as a delimiting factor to identify phases in interactive applications.
Each application has different behaviors with user interactions through touch screen, and we show
how TLP and utilization change differently for various application phases.

Finally, we show how the state-of-art power management system in smartphones behaves with 
the interactive applications. Such a power management system available in the current smartphone operating system
attempts to minimize wasted cycles with DVFS. At the same time, the power management system must not affect the users' perception
on system performance and interactivity. However, since the current multi-cores for smartphones do not support per-core
DVFS due to the area limitation, the power management system may miss the opportunity to further 
reduce the power consumption. This paper explores where the further opportunities for energy reduction exist with
multi-core platforms and interactive applications.


