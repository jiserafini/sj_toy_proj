%\vspace{-0.2in}
\section{Conclusions}
%\vspace{-0.2in}

This paper characterizes the power behaviors of smartphone multi-cores and interactive applications. 
Although the number of cores are increasing, the available TLP is still limited in the interactive
applications. Furthermore, the available TLP is highly skewed, with one or two threads using
a significant portion of CPU utilization. We classified main applications phases into
interactive and responsive phases, and showed that the two phases have different usage patterns
and responsive requirements.
Based on the analysis, we proposed a power management technique with three components,
dominant thread scheduling, phase-aware headroom adjustment, and thread packing. The interactive
applications benefit most from the phase-aware headroom adjustment due to their stability
of CPU usages. Although the proposed technique does not save power significantly for
the responsive applications, the scheduling scheme reduces the execution delay significantly,
and thus improves energy and $ED^2$ for the responsive applications. 



