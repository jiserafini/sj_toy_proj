\section{Introduction}

Although smartphones dominate mobile platforms, 
the primary factor limiting their mobility is the energy efficiency and thus the battery 
life. Among the components in the smartphones, processors have been considered to be one of the top two power-hungry
components, along with the display device~\cite{smartphonepower,intothewild}. Meanwhile, the processors for smartphones have 
evolved into multi-cores, making the exploitation of thread-level parallelism (TLP) critical 
for the performance and energy efficiency~\cite{desktopTLP}. To fully utilize the potential benefit of multi-cores, and 
to improve the energy efficiency, it is critical to understand the behaviors of smartphone applications for their
available TLP. 

However, such smartphone applications exhibit quite different behaviors from the traditional CPU benchmark applications
used for servers and desktops. Their executions are highly dependent upon the interactions with users, and the
power consumption behaviors are also affected by user interactions. Although the importance of understanding such
interactive applications has been growing, there  have been a limited number of studies on the behavior of 
smartphone applications~\cite{intothewild, characterinteractive}. 
In this paper, we first characterize the parallelism and CPU utilization behaviors of smartphone applications.
Using a commercial smartphone platform with a quad core processor, 
this paper investigates the thread-level parallelism (TLP) and 
CPU utilization, which are directly related to the power management with dynamic voltage and frequency scaling (DVFS) and dynamic core plug-ins~\cite{packandcap}.

To characterize the TLP behavior, we use user interaction as a delimiting factor to identify phases 
in interactive applications. We classify execution phases into two main phases, {\it interactive}, 
and {\it responsive}.
In the interactive phase, the application continuously uses some CPU resources, maintaining
some level of CPU utilization, while the user may change
the execution path. In the responsive phase, the application uses CPU resources for a short period
time only after a user input.
Although an application can exhibit both interactive and responsive phases, most of the common
mobile applications are dominated by only one of the two phases.

Although the two phases exhibit different CPU usage patterns, TLP is still limited in
mobile platforms with a relative small number of active threads. Furthermore, the CPU utilization 
is also often much lower than the maximum CPU capability with the highest clock frequency. 
Among the active threads, applications often exhibit the heterogeneity of threads, with a thread
accounting for a significant portion of the overall CPU utilization, which we call a {\it dominant thread}.
Although many threads become active during interactive or responsive phases, the majority of them
use CPU for a very short period of time. Only one or two thread per application 
maintain CPU utilization at a certain level.

Since the CPU utilization is relatively low in interaction-oriented mobile applications, the 
current DVFS-based power management scheme reduces clock frequencies to minimize unused cycles. 
At the same time, the power management scheme must not affect the users' perception on system performance and 
interactivity. Therefore, the current power management scheme, without considering the requirement
of different phases, often uses a conservative frequency management with a large headroom 
for CPU computational capacity. The current management schemes determine the number of active cores
solely based on the utilization level, without considering the number of active threads, and the 
existence of dominant threads.

Based on the TLP and interaction analysis for different phases of mobile applications, the paper proposes an improved
DVFS-based power management. The proposed power management scheme uses three techniques,
{\it dominant thread (DT) aware scheduling}, {\it phase-aware headroom adjustment}, and {\it thread packing} 
to improve energy efficiency without incurring any perceived performance 
degradation for mobile applications.
The DT-aware scheduling identifies a dominant thread and prioritizes it to a dedicated core, since
the performance of the dominant thread often determines the overall interactivity perceived by the user.
The phase-aware headroom adjustment dynamically determines the frequency headroom when
the clock frequency is changed for adjusting CPU capability. Depending on phase classification,
the headroom is adapted to minimize the wasted CPU capacity, instead of using a fixed headroom
for the current conservative schemes.
The thread packing finds the optimal number of active cores, and assign threads to maximize
the energy efficiency. After assigning a dominant thread to a fixed core,
the rest of non-dominant threads are packed to an optimal number of active cores with the best energy
efficiency. 

The main contributions of this paper are as follows.

\begin{itemize}

\item This paper presents an in-depth analysis of thread-level parallelism in a popular mobile platform. Using
phase classification, the paper provides available parallelism and CPU utilization characteristics of threads, showing
the existence of dominant threads in common mobile applications. This paper is one of the first studies to analyze 
the TLP of interactive mobile applications.

\item The paper proposes a TLP-aware scheduling scheme to improve the energy efficiency of mobile platforms. The proposed
schemes exploit the heterogeneous CPU utilization behaviors among active threads, and prioritize a dominant thread.
Based on a power model, it identifies the best number of active cores, packing or unpacking non-dominant threads
to the cores.

\item The paper proposes a dynamic adjustment of frequency headroom used by the power governor  to reduce unnecessary 
headroom for DVFS. Unlike the prior scheme, the dynamic adjustment of headroom minimizes unnecessary increases
of clock frequency to improve system responsiveness.

\end{itemize}

The rest of the paper is organized as follows.
Section 2 presents our analysis of power behaviors of mobile platforms and current DVFS-based power management schemes.
Section 3 characterizes the TLP behaviors and shows the dominant thread execution in a set of mobile applications.
Section 4 presents an improvement power management with TLP-aware scheduling and phase-aware headroom adjustment.
Section 5 presents the experimental results, and Section 6 concludes the paper.


%This paper characterizes three different aspects of power behaviors in commercial smartphone platforms and 
%their applications. Firstly, this study analyzes the actual power consumption of multi-cores used for
%smartphones with varying clock frequencies and CPU utilization. We used two types of micro-benchmarks,
%one for stressing CPUs and the other for stressing the memory system. The analysis shows how the 
%current multi-cores and their support for dynamic voltage and frequency scaling (DVFS) consume power 
%with the two extreme cases of application types.

%Gutierrez et al. showed the characteristics of smartphone applications, focusing their micro-architectural
%aspects~\cite{characterinteractive}. 
