\section{Improving Power Management for Mobile Platforms}

In this section, we propose three techniques to improve the energy efficiency of
mobile platforms, exploiting the TLP characteristics and interaction types of
applications. The first technique identifies a dominant thread and
assign a dedicated core to the dominant thread, to preserve the context
in the cache, and avoid any unnecessary interference with other threads.
The second technique attempts to maximize the utilization by lowering
the clock frequency as much as possible. To avoid any performance degradation,
it exploits the usage difference between interactive and responsive phases.
The third technique determines the most energy efficient number of
active cores for a given set of current active threads. Using a utility-based
power model, it picks the number of cores, and assign non-dominant threads
evenly to the cores.
Algorithm~\ref{algo:tlpdvfs} shows the proposed power management with the 
three techniques ({\tt part 2, part 3, and part 4} ).

\subsection{Dominant Thread Scheduling for Energy Efficiency}

The first technique identifies and prioritizes the execution of
a dominant thread. As shown in Section 3.3, 
common mobile applications have a dominant thread, and often,
the performance perceived by the user is determined by the delay of 
the dominant thread. Not to degrade the user perceived performance,
any power management scheme must not increase the delay of
the dominant thread. Furthermore, reducing the delay of a dominant thread
can potentially reduce the CPU energy by completing a phase as early
as possible.

In the proposed power management scheme, a dominant thread is identified
with the energy management module, tracking the CPU utilization of 
threads dynamically. Once a dominant thread is identified, the thread
is scheduled to a fixed core. Assigning a fixed core to a dominant thread
has several benefits. Any interference, including the cache pollution, 
by sharing a core with other minor threads can be avoided. Furthermore, since the
dominant thread account for a significant portion of CPU utilization,
a migration of the dominant thread from a core to another can fluctuate
the utilization of cores, which makes utilization less stable.
In Algorithm~\ref{algo:tlpdvfs}, {\tt part 2} shows the dominant thread scheduling
({\tt DT-aware scheduling}).

%\begin{algorithm}[t]
\begin{algorithm}[t]
\caption{TLP-aware DVFS and scheduling}
\label{algo:tlpdvfs}
\begin{algorithmic}
{\footnotesize
\STATE // initially, $headroom \gets 10$,
\STATE // $inc\_headroom \gets 0$, $dec\_headroom \gets 0$
\STATE
\STATE // part 1: check the usefulness of sample
\IF {$core\_util == 100\%$ for any core}
\STATE set the maximum frequency
\STATE set\_scheduler\_timer(10ms)
\RETURN
\ENDIF
\STATE
\STATE // part 2: DT-aware scheduling
\FOR {each thread}
\STATE {\bf if} $thread\_util >= DOMINANT\_THRESHOLD$ %XXX
\STATE \quad mark as a dominant thread
\ENDFOR
\STATE allocate one core for each dominant threads
\STATE
\STATE // part 3: Adjust headroom 
\STATE $freq\_util \gets \sum_{i=0}^{num\_cores} core\_freq_i * core\_util_i$
\STATE $history$.append($freq\_util$)
\STATE $stdev$ = standard deviation of the last 16 entries of $history$
\STATE {\bf if} $stdev > LARGE\_STDEV$
\STATE \quad {\bf then} $inc\_headroom++$; $dec\_headroom \gets 0$
\STATE {\bf else if} $stdev < SMALL\_STDEV$
\STATE \quad {\bf then} $dec\_headroom++$; $inc\_headroom \gets 0$
%\STATE {\bf else}
%\STATE \quad {\bf then} $dec\_headroom \gets 0$; $inc\_headroom \gets 0$
\STATE
\STATE {\bf if} $inc\_headroom > INC\_HEADROOM\_THRESHOLD$
\STATE \quad {\bf then} $headroom++$
\STATE {\bf if} $dec\_headroom > DEC\_HEADROOM\_THRESHOLD$
\STATE \quad {\bf then} $headroom--$
\STATE
\STATE // part 4: thread packing
\STATE $best\_core \gets 0$, $best\_freq \gets 0$
\STATE $min\_power \gets VERT\_LARGE\_NUMBER$
\FOR {$i \gets 1 ... num\_cores$}
\STATE Divide $thread\_util$s to $i$ cores to as evenly as possible
%\STATE $largest\_util \gets$ largest utilizatoin among cores
\STATE $freq \gets$ the smallest frequency the largest utilization
\STATE \quad\quad\quad\quad will be less than $(100-headroom)$
\STATE $power \gets$ power\_model($freq$, $thread\_util$ distribution)
\IF {$power < min\_power$}
\STATE $best\_core \gets i$, $best\_freq \gets freq$
\STATE $min\_power \gets power$
\ENDIF
\ENDFOR
\STATE
\STATE pack threads to $best\_core$ and set the frequency as $best\_freq$
\STATE set\_scheduler\_timer(100ms)
\RETURN
}
\end{algorithmic}
\end{algorithm}


\subsection{Dynamic Headroom Adjustment}

\begin{figure}[bt]
\begin{center}
\epsfig{figure=graphs/app_stddev.eps, width=7cm}
\vspace{-0.2in}
\end{center}
\caption{Standard Deviation Difference}
\label{fig:stdv_phase}
\end{figure}

The second technique adjusts the headroom for DVFS. In the original on-demand 
governor as shown in Algorithm~\ref{algo:ondemand}, when the current utilization is higher than
{\tt up-threshold}, the frequency is increased to the max frequency to enhance the user responsiveness.
However, it can unnecessarily increase the frequency, as even from the lowest 200MHz,
the frequency can be increased to the peak frequency.

The second technique attempts to maintain the highest possible utilization by
setting the clock frequency as low as possible. Unlike the on-demand governor 
policy, the clock frequency is increased by steps, even at high utilization.
However, a problem with such
a narrow margin in utilization is that it may degrade performance when the
utilization fluctuate significantly. To mitigate the effect, the scheme adds 
a small percentage of headroom when it sets the clock frequency.

The headroom for clock frequency setting is adjusted dynamically.
The interactive and responsive phases have different requirements for the 
frequency headroom. The interactive phases tend to have stable utilizations regardless
of user inputs, so they do not require a high headroom, allowing to reduce 
frequency at the lowest level. 
However, responsive phases exhibit high fluctuation of utilization,
and thus a large headroom is required not to degrade user responsiveness for a sudden surge
of CPU utilization. Figure~\ref{fig:stdv_phase} presents the standard deviation of
application loads for the two classes of applications. The figure shows that
the interactive applications exhibit much less fluctuation of utilization, and
such stable utilization allows a small headroom for frequency setting.

\begin{comment}
\begin{table}[tb]
\begin{center}
%\begin{footnotesize}
\begin{tabular}{r|r}
\hline \hline
Threshold       & Probability   \\ \hline
80      &       0.003   \\
85      &       0.013   \\
90      &       0.003   \\
95      &       0.009   \\
\hline \hline
\end{tabular}
%\end{footnotesize}
\end{center}
%\vspace*{-0.15in}
\caption{Probability of peak utilization with the interative applications}
\label{tab:prob_peak}
%\vspace*{-0.2in}
\end{table}

To show the utilization stability of interative applications, Table~\ref{tab:prob_peak} presents
the probability of peak utilization for the interative applications.
It checks whether CPU utilization reaches 99-100\%, meaning the current
frequency setting can be too low for the current CPU load. Such a low frequency setting can degrade the application
performance, as the application cannot receive enough CPU resource.
It uses the on-demand governor, and uses four different {\tt up-threshold} settings from the conservative 80\% 
to the most aggresive 95\%. As shown by the table, the utility of interactive applications do not require
the conservative policy by the on-deman governor, as with even with a very high threadhold of 95\%, the probability
\end{comment}

As shown in Algorithm~\ref{algo:tlpdvfs} ({\tt part 2}),
the power manager maintains the past 16 samples of utilization.
If the variation of the utilization samples is higher than a threshold, the headroom
is increased, expecting responsive phases. If the variation stays low, the headroom
is decreased to reduce unnecessary frequency increase.

\subsection{Thread Packing}

The final technique is to determine the most energy efficient number of active
cores, and pack threads to the determined number of cores. A dominant thread is
scheduled to a separate core, and the other non-dominant threads are packed 
into the rest of cores. 
Finding an optimal number of cores and packing threads on the cores have been
proposed by Cochran et al. \cite{packandcap}  However, our method uses 
a much simpler power model, since a utility-based model is good enough for
mobile applications, unlike compute-intensive multi-threaded applications used
by the prior study.

To determine the optimal number of cores, we use the utility-based power model, as discussed
in Section 2.2. Using the power model, for the current utilization for active threads,
the best number of cores and frequency is selected by searching all possible combinations
of power for different frequency and core counts. In {\tt part 4} of Algorithm~\ref{algo:tlpdvfs},
the power model is used to select the best number of cores and the minimum frequency 
to satisfy the current utilization with the frequency headroom determined by the headroom
adjustment ({\tt part 2})
The proposed technique can be used regardless of the support for per-core DVFS. 

