\subsection{Methodology}
In this section, we describe our experimental methodology. 
Our target device is a smart phone which has the Exynos4412 CPU with four cores.
The device runs the Android 4.0.4 operating system which includes Dalvik Virtual Machine (VM). 
All applications run on an instance of their Dalvik VM. 
The Moonsoon power meter measures the power consumption for the whole system that uses a single lithium (Li) battery. 
To evaluate the effectiveness of our power management scheme, we use power, energy, and $ED^2P$ (Energy*Delay*Delay) metrics to assess
the differnt aspects of the proposed schemes.

Although, for the benchmark application shown in Table~\ref{tab:app_summary}, we measure the power consumption, the execution
delay is measured only for the responsive applications. The execution time of an interactive application is determined by
how long a user uses the application, and it is not directly affected by CPU performance. Instead, if the performance
is degraded, the user will perceive the slowdown of the execution, for example slow changes of game scenes etc.
In the paper, to show that the proposed scheme does not incur any perceivable performance degradation, we 
conduct a user study with a 13 user sample for the four benchmark applications with interactive phases.

For the responsive applications, we define a delay as the time period from the initation of a user input to the completion
of the initiated job. As the delay is shortened, the processor can become idle sooner, reducing the CPU energy.
To measure the delay for the web browser and mobile chrome, 
we use BBench which is a web-page rendering benchmark with 11 of the popular sites such as Amazon, BBC, and eBay etc \cite{characterinteractive}. 
The BBench shows page rendering time for each of the pages. 
To measure the delay for the other responsive applications in a repeatable manner, we use the MonkeyRunner tool which 
provides the emulation and automation of diverse user actions such as touch, drag, etc \cite{monkeyrunner}.
The Google Maps for the android application selects search button, types a popular city address, and get directions. 
The search is repeated 5 times with different city addresses. 
The photo editor application selects one of photos, hits the touch effect button, and apply different 12 effects. 
The ezPDF Reader opens different 8 files with the pdf file format.

We evaluate three configurations, including dominant thread scheduling ({\tt DT scheduling}), 
phase-aware headroom adjustment ({\tt phase-aware DVFS}), and thread packing ({\tt packing}). The baseline system uses
the on-demand governor, and all the results are normalized to those with the baseline configuration. 

%we start the real mobile application applied the MonkeyRunner which emulated user diverse action and automated a series of user interaction
%for example, typing a specific city address, touching a button to enter next step, and dragging the touch screen. 
%Third, the application runs by user interaction, step by step, when CPUs become idle state. Finally, 
%all of interactive action is finished and display execution delay of the application.

