\chapter{Experimental Results}

\section{Methodology}
In this section, we describe our experimental methodology. 
Our target device is a state-of-a-art smart phone which has the Exynos4412 CPU with four cores.
The device runs the Android 4.0.4 operating system which includes Dalvik Virtual Machine (VM). 
All applications run on an instance of their Dalvik VM. 
The Moonsoon power meter measures the power consumption for the whole system that uses a single lithium (Li) battery. 
To evaluate the effectiveness of our power management scheme, we use power, energy, and $ED^2P$ (Energy*Delay*Delay) metrics to assess
the differnt aspects of the proposed schemes.

Although, for the benchmark application shown in Table~\ref{tab:app_summary}, we measure the power consumption, the execution
delay is measured only for the responsive applications. The execution time of an interactive application is determined by
how long a user uses the application, and it is not directly affected by CPU performance. Instead, if the performance
is degraded, the user will perceive the slowdown of the execution, for example slow changes of game scenes etc.
In the paper, to show that the proposed scheme does not incur any perceivable performance degradation, we 
conduct a user study with a 13 user sample for the four benchmark applications with interactive phases.

For the responsive applications, we define a delay as the time period from the initation of a user input to the completion
of the initiated job. As the delay is shortened, the processor can become idle sooner, reducing the CPU energy.
To measure the delay for the web browser and mobile chrome, 
we use BBench which is a web-page rendering benchmark with 11 of the popular sites such as Amazon, BBC, and eBay etc \cite{characterinteractive}. 
The BBench shows page rendering time for each of the pages. 
To measure the delay for the other responsive applications in a repeatable manner, we use the MonkeyRunner tool which 
provides the emulation and automation of diverse user actions such as touch, drag, etc \cite{monkeyrunner}.
The Google Maps for the android application selects search button, types a popular city address, and get directions. 
The search is repeated 5 times with different city addresses. 
The photo editor application selects one of photos, hits the touch effect button, and apply different 12 effects. 
The ezPDF Reader opens different 8 files with the pdf file format.

We evaluate three configurations, including dominant thread scheduling ({\tt DT scheduling}), 
phase-aware headroom adjustment ({\tt phase-aware DVFS}), and thread packing ({\tt packing}). The baseline system uses
the on-demand governor, and all the results are normalized to those with the baseline configuration. 

%we start the real mobile application applied the MonkeyRunner which emulated user diverse action and automated a series of user interaction
%for example, typing a specific city address, touching a button to enter next step, and dragging the touch screen. 
%Third, the application runs by user interaction, step by step, when CPUs become idle state. Finally, 
%all of interactive action is finished and display execution delay of the application.



\section{Power}

\begin{figure}[bt]
\begin{center}
\epsfig{figure=graphs/interactive_power_result.eps, width=8cm}
%\vspace{-0.2in}
\end{center}
\caption{Power saving percentage normalized to Ondemand governor}
\label{fig:interactive_power_result}
\end{figure}

\begin{figure}[bt]
\begin{center}
\epsfig{figure=graphs/responsive_power_result.eps, width=8cm}
%\vspace{-0.2in}
\end{center}
\caption{Power saving percentage normalized to Ondemand governor}
\label{fig:responsive_power}
\end{figure}

We first evaluate the power saving with the proposed techniques. Figure~\ref{fig:interactive_power_result} and~\ref{fig:responsive_power} 
present the power saving compared to the baseline on-demand governor, for the interactive, and responsive applications respectively. 
For the interactive applications, the proposed techniques effectively reduce power up-to 20\% for {\tt angry bird}. 
The moive player has little improvement, since the on-demand governor can set the frequency effectively for the application with
a low and stable utilization. Among the three techniques, the interactive applications benefit most from the phase-aware headroom
adjustment. Since the utilization is relatively stable, the tight frequency setting can reduce the overall power, without
hurting user interactivity, as will be discussed with a user study in Section 5.6. 

For the responsive applications, three techniques have a relatively small power saving, or increase power for the two applications.
The headroom adjustment cannot reduce the headroom for the responsive applications effecitvely, since they have a high fluctuation of 
utlization, and thus require a large headroom not to degrade the performance and system responsiveness. 

\section{Delay, Energy, and ED\textsuperscript{2}P for the responsive applications}

\begin{figure}[bt]
\begin{center}
\epsfig{figure=graphs/responsive_latency_result.eps, width=8cm}
%\vspace{-0.2in}
\end{center}
\caption{Latency Improvement percentage normalized to Ondemand governor}
\label{fig:responsive_latency_result}
\end{figure}

\begin{figure}[bt]
\begin{center}
\epsfig{figure=graphs/responsive_energy_result.eps, width=8cm}
%\vspace{-0.2in}
\end{center}
\caption{Energy saving percentage normalized to Ondemand governor}
\label{fig:responsive_energy_result}
\end{figure}

Although the proposed scheme does not provide large power saving for the responsive applications, DT-ware scheduling and
thread packing is quite effective in reducing the delay. Figure~\ref{fig:responsive_latency_result} shows 
the normalized delay for the responsive applications. Among the three techniques, fixing a dominant thread to
a core reduces the delay significantly for the responsive applications. A dominant thread can use
caches without any pollution from other threads, and do not need to be migrated between cores.
Figure~\ref{fig:responsive_energy_result} shows the energy saving for the responsive applications. 
Since the delay is significantly reduced by the proposed scheduing scheme, the actual energy consumption
is reduced, even if the power may increase slightly as shown in Figure~\ref{fig:responsive_power}.
The overall energy saving can be from 5\% to 10\% across the 5 responsive applications.

\begin{figure}[bt]
\begin{center}
\epsfig{figure=graphs/responsive_EDP_result.eps, width=8cm}
%\vspace{-0.2in}
\end{center}
\caption{ED\textsuperscript{2}P Improvement percentage normalized to Ondemand governor}
\label{fig:responsive_EDP_result}
\end{figure}

Figure~\ref{fig:responsive_EDP_result} shows the ED\textsuperscript{2}P improvements by the proposed technique.
For the responsive applications, the fast system response is critical for user experience in mobile platforms,
as the increasing computational power with multi-cores is supposed to improve the system responsiveness.
In the figure, ED\textsuperscript{2}P improvements are signifcant from 10-33\% for the four applications,
except for the pdf reader.

\section{System Reponsiveness}

\begin{figure}[bt]
\begin{center}
\epsfig{figure=graphs/interactive_UX_result.eps, width=8cm}
%\vspace{-0.2in}
\end{center}
\caption{User satisfaction test normalized to Ondemand governor}
\label{fig:interactive_UX_result}
\end{figure}

To assess the system responsiveness for the four interactive applications, we conduct a user study.
We asked 13 users to evaluate the effect of our techniques on the user satisfaction. 
A volunteer uses applications with ondemand governor, assuming the governor satisfy all users, and experience the application execution first 
with the baseline governor. 
After that, the user tries the same application with the proposed governor without any notification 
to identify whether the volunteers detect potential poor interactivity.  
Figure~\ref{fig:interactive_UX_result} shows the results of our user study for four applications. 
Only one user reports a minor slowdown for a game, among 13 users for four applications.
The results show many users do not sense the performance loss for the majority of applications.

\begin{comment}
\begin{figure}[bt]
\begin{center}
\epsfig{figure=graphs/latency_different_headroom_size.eps, width=7cm}
\vspace{-0.2in}
\end{center}
\caption{Latency between user input and frequency raising}
\label{fig:latency_different_headroom_size}
\end{figure}

The second method for system responsiveness is to check whether CPU utilization reaches 99-100\%, meaning the current
frequency setting can be too low for the current CPU load. Such a low frequency setting can degrade the application 
performance, as the application cannot receive enough CPU resource.

\end{comment}
