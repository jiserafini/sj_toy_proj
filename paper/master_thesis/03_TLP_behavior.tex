\chapter{TLP Behavior of Mobile Applications}

\section{Phase Classification}

In mobile applications, the user interactions dominate the CPU usage patterns, and
affect the TLP and utilization significantly. Although each application exhibits
different types of user interactions and CPU usage patterns, we broadly classify
the execution phases of mobile applications into three phases, {\tt setup},
{\tt responsive}, and {\tt interactive}.
Firstly, a {\tt setup} phase initializes an application, or loads data for a new round 
of game. In this period, user interactions do not affect the CPU usage significantly.
Secondly, a {\tt responsive} phase occurs when cores become significantly active only after a user interaction.
In such a responsive phase, CPUs become idle or show low utilization sometime after the interaction, after
the computation is completed to process the user interaction. A common example of
the responsive phase is the interaction with a web browser application. Once
a hyperlink is touched, the CPUs become active to render a new page. However,
once the process is completed, the CPUs become idle again.

Finally, in an {\tt interactive} phase, the application continuously conducts some work
while users may interact with the application. For games, 
CPUs process the game engine continuously regardless of user interactions. At the same time,
user interactions may change the behaviors of such computational work.
However, even in an interactive phase, the CPU is not completely occupied with 
active threads, and the CPU utilization do not change significantly.
An application may exhibit only a subset of the three phases. For example, some applications
may not have any responsive phases.

\begin{figure}[bt]
\begin{center}
\epsfig{figure=graphs/angrybird_interactive.eps, width=8cm}
%\vspace{-0.45in}
\epsfig{figure=graphs/browser_1cpu.eps, width=8cm}
%\vspace{0.01in}
\end{center}
\caption{CPU util. and touch events with 1 core}
%\vspace{-0.2in}
\label{fig:interact_util_1cpu}
\end{figure}

To show the detailed behaviors of applications, Figure~\ref{fig:interact_util_1cpu} presents
the utilization of a single 1GHz core for a game application (Angry bird) and a web browser.
The dots represent user touch inputs on the touch screen.
For the game application, interactive phases dominate the execution.
In an interactive phase,
CPU utilization is maintained around 40\%, regardless of user interactions.
During the interactive phase, even if the user affects the application execution though
touch inputs, the CPU utilization is dominantly determined by the game engine routine.
However, in the browser application, {\tt response} phases are mostly
shown. CPU utilization increases only when a user touch occurs, and the utilization
drop soon after the interaction is processed.

\begin{table}[tb]
\begin{center}
\begin{footnotesize}
\begin{tabular}{l|l|l}
\hline \hline
Main Phase & App & Description \\
\hline
\multirow{4}*{Interactive}	& Trial Winter		& 3D motorcycle racing game	\\ \cline{2-3}
& Eternity Warriors 	& 3D action RPG game		\\\cline{2-3}
& Angry Brid 		& Shooting game with physics engine \\\cline{2-3}
& Movie Player		& Play a video file \\\cline{2-3}
\hline
\multirow{5}*{Responsive}	&Photo Editor		& Apply various filter to the photo \\\cline{2-3}
& Google Map			& Search a place and magnify \\\cline{2-3}
& ezPDF Reader		& Read the pdf document \\\cline{2-3}
& Browser				& BBench on default browser \\\cline{2-3}
& Mobile Chrome		& BBench on Chrome browser \\
\hline \hline
\end{tabular}
\end{footnotesize}
\end{center}
%\vspace*{-0.15in}
\caption{Test Application Set}
%\vspace*{-0.2in}
\label{tab:app_summary}
\end{table}


Although applications have all or a subset of the three phases, one of the three phases
dominate each application. For example, for the majority of game applications, 
the interactive phase appears frequently. Web browsers are commonly dominated by
responsive phases. Table~\ref{tab:app_summary} shows a set of benchmark smartphone 
applications classified by their main dominant phase.
The first four applications are games and a movie player, and the applications
have all three phases. However, for the four applications, interactive phases
dominate the execution time of CPUs.
In the next five applications, responsive phases dominate application execution. In those applications,
interactive phases do not appear, since the most of CPU usage is initiated by user inputs.

\begin{table}[tb]
\begin{center}
\begin{footnotesize}
\begin{tabular}{l|l|r|r}
\hline \hline
App		& Metric	& w/o touch		& w/ touch \\
\hline
\multirow{2}*{Trial Winter}	& CPU util.(\%)	& 68.93	& 76.39	\\
							& \# active threads	& 1.36	& 1.34	\\
\hline
\multirow{2}*{Eternity Warriors} & CPU util.(\%)	& 87.44	& 93.19	\\
							& \# active threads	& 1.49	& 1.88	\\
\hline
\multirow{2}*{Angry Brid}	&  CPU util.(\%)	& 41.92	& 48.14	\\
					& \# active threads & 0.64	& 1.43	\\
\hline
\multirow{2}*{Movie Player}		&  CPU util.(\%)	& 22.78	& 25.21	\\
					& \# active threads	& 0.20	& 0.83 \\
\hline \hline
\multirow{2}*{Photo Editor}		&  CPU util.(\%)	& 2.22	 & 73.42	\\
					& \# active threads	 & 0.03	& 3.42	\\
\hline
\multirow{2}*{Google Map}		& CPU util.(\%)	& 23.12& 86.69 \\
					& \# active threads	& 0.46	& 2.06	\\
\hline
\multirow{2}*{ezPDF Reader}	& CPU util.(\%)	& 1.45	& 43.71	\\
					& \# active threads & 0.03	& 1.04 \\
\hline
\multirow{2}*{Browser} 	& CPU util.(\%)	& 12.23	& 86.92	\\
					& \# active threads & 0.22	& 4.21	\\
\hline
\multirow{2}*{Mobile Chrome}	& CPU util.(\%)	& 0.58	& 95.05	\\
					& \# active threads & 0.03	& 2.43	\\
\hline \hline
\end{tabular}
\end{footnotesize}
\end{center}
%\vspace*{-0.15in}
\caption{Summary of Application Phase Characteristics (One core 1.4GHz)}
%\vspace*{-0.2in}
\label{tab:tlp_summary}
\end{table}


\section {Available TLP}

In this section, we present the CPU utilization and available TLP during the main phase
of each type of applications. 
Table~\ref{tab:tlp_summary} presents the utilization and TLP for our benchmark applications in
their main phase, the first four with the interactive phase, and the rest with
the responsive phase.

For each phase of an application, two numbers are shown. Firstly,
CPU utilization is an average utilization for one core with 1.4Ghz during the phase. 
The second row, the number of
active threads presents the available parallelism. 
A phase is further decomposed into a portion of the phase with and without
user interactions. 

Firstly, the overall utilization is relatively low for mobile applications. The utilization is
only measured with one core with 1.4Ghz. Considering four available cores in the target platform,
even a single core is not fully utilized. For games, the CPU utilization is between 48-93\%
with a core, but the movie player exhibits much lower 25\% utilization. For interactive
applications, touch inputs do not affect the CPU utilization significantly. However, as expected,
the utilizations for responsive applications are high only with touch inputs. 

Secondly, the average number of active threads during the main phase is quite limited. Most
of applications have only 1-2 active threads at any given time to be scheduled to cores. 
An exception is the web browser which has more than four active threads. For such applications
with several threads, multiple cores can be used with much lower clock frequencies to reduce
the overall energy.

\section{Dominant Thread}

\begin{table}[tb]
\begin{center}
\begin{footnotesize}
\begin{tabular}{l|l|r|r}
\hline \hline
App		& Metric	& w/o touch		& w/ touch \\
\hline
\multirow{2}*{Trial Winter}	& Avg. portion(\%)	& 44.67	& 44.92	\\
						& \# of exec thread & 55 & 55 \\
\hline						
\multirow{2}*{Eternity Warriors}	& Avg. portion(\%)	& 33.65	& 37.73	\\
					& \# of exec thread & 103 & 107 \\
\hline
\multirow{2}*{Angry Brid}			& Avg. portion(\%)	& 35.14	& 37.46	\\
					& \# of exec thread & 50 & 53 \\
\hline
\multirow{2}*{Movie Player}	& Avg. portion(\%)	& \multicolumn{1}{c|}{-}	& \multicolumn{1}{c}{-}	\\
					& \# of exec thread & 30  & 34 \\
\hline	\hline						
\multirow{2}*{Photo Editor}	& Avg. portion(\%)	&\multicolumn{1}{c|}{-}	& 53.42\\
					& \# of exec thread &  \multicolumn{1}{c|}{-} & 138 \\
\hline
\multirow{2}*{Google Map}	& Avg. portion(\%)	& \multicolumn{1}{c|}{-}	& 25.55	\\
					& \# of exec thread & \multicolumn{1}{c|}{-} & 275 \\
\hline
\multirow{2}*{ezPDF Reader}		& Avg. portion(\%)	& \multicolumn{1}{c|}{-}	& 34.96	\\
					& \# of exec thread & \multicolumn{1}{c|}{-} & 65 \\
\hline
\multirow{2}*{Browser}	& Avg. portion(\%)	& \multicolumn{1}{c|}{-}	& 29.12	\\
					& \# of exec thread & \multicolumn{1}{c|}{-} 	&  285 \\
\hline
\multirow{2}*{Mobile Chrome}	& Avg. portion(\%)	& \multicolumn{1}{c|}{-}	& 38.16	\\
					& \# of exec thread & \multicolumn{1}{c|}{-}	& 89\\
\hline	\hline				
\end{tabular}
\end{footnotesize}
\end{center}
%\vspace*{-0.15in}
\caption{Analysis of Dominant Thread (Four cores 1GHz)}
\label{tab:dom_threads}
%\vspace*{-0.2in}
\end{table}

\begin{comment}
\begin{table}[tb]
\begin{center}
\begin{footnotesize}
\begin{tabular}{l|l|r|r}
\hline \hline
App		& Metric	& w/o touch		& w/ touch \\
\hline
\multirow{3}*{Trial Winter}	& Avg. portion(\%)	& 44.67	& 44.92	\\
						& \# of exec thread & 55 & 55 \\
						& Avg. relation & 1.00	& 1.00	\\
\hline						
\multirow{3}*{Eternity Warriors}	& Avg. portion(\%)	& 33.65	& 37.73	\\
					& \# of exec thread & 103 & 107 \\
					& Avg. relation & 0.95	& 0.93	\\
\hline
\multirow{3}*{Angry Brid}			& Avg. portion(\%)	& 35.14	& 37.46	\\
					& \# of exec thread & 50 & 53 \\
					& Avg. relation	& 1.00	& 1.00	\\
\hline
\multirow{3}*{Movie Player}	& Avg. portion(\%)	& \multicolumn{1}{c|}{-}	& \multicolumn{1}{c}{-}	\\
					& \# of exec thread & 30  & 34 \\
					& Avg. relation 	& \multicolumn{1}{c|}{-}	& \multicolumn{1}{c}{-}	\\
\hline	\hline						
\multirow{3}*{Photo Editor}	& Avg. portion(\%)	&\multicolumn{1}{c|}{-}	& 53.42\\
					& \# of exec thread &  \multicolumn{1}{c|}{-} & 138 \\\
					& Avg. relation		&\multicolumn{1}{c|}{-}		& 0.67	\\
\hline
\multirow{3}*{Google Map}	& Avg. portion(\%)	& \multicolumn{1}{c|}{-}	& 25.55	\\
					& \# of exec thread & \multicolumn{1}{c|}{-} & 275 \\
					& Avg. relation 	& \multicolumn{1}{c|}{-}	& 0.59	\\
\hline
\multirow{3}*{ezPDF Reader}		& Avg. portion(\%)	& \multicolumn{1}{c|}{-}	& 34.96	\\
					& \# of exec thread & \multicolumn{1}{c|}{-} & 65 \\
					& Avg. relation 	& \multicolumn{1}{c|}{-}	& 0.66	\\
\hline
\multirow{3}*{Browser}	& Avg. portion(\%)	& \multicolumn{1}{c|}{-}	& 29.12	\\
					& \# of exec thread & \multicolumn{1}{c|}{-} 	&  285 \\
					& Avg. relation 	& \multicolumn{1}{c|}{-} 	& 0.56	\\
\hline
\multirow{3}*{Mobile Chrome}	& Avg. portion(\%)	& \multicolumn{1}{c|}{-}	& 38.16	\\
					& \# of exec thread & \multicolumn{1}{c|}{-}	& 89\\
					& Avg. relation 	& \multicolumn{1}{c|}{-}	& 0.72	\\
\hline	\hline				
\end{tabular}
\end{footnotesize}
\end{center}
%\vspace*{-0.15in}
\caption{Analysis of Dominant Thread (four 1GHz cores)}
\label{tab:dom_threads}
%\vspace*{-0.2in}
\end{table}
\end{comment}



\begin{figure}[relative_util]
\begin{center}
\epsfig{figure=graphs/relative_util_normalized_min_exec_time.eps, width=8cm}
%\vspace{-0.45in}
\end{center}
\caption{Relative Util. Normalized to Minimum Thread Exec. Time}
%\vspace{-0.2in}
\label{fig:relative_util}
\end{figure}



To minimize the power for multi-cores, more cores should be used with the lowest possible frequency.
However, it must be able to accommodate the active threads and their utilization requirements. 
Another important aspect of TLP is how each thread consumes CPU resources. Although there is 
a limited number of active threads in mobile applications during their main phases, the CPU utilizations
of the threads are quite skewed. One or two threads dominate the overall CPU utilization, even though
there are many other threads, which shortly use CPU.

Table~\ref{tab:dom_threads} presents the behaviors of dominant threads in our benchmark applications.
For each application, the first line shows the average portion of utilization of a dominant thread during
the main phase. For example, for angry bird, a single dominant thread uses 37\% of overall CPU utilization 
during the phase. The second line shows the number of threads, which are activated at least once 
during the main phase. The number of
executed threads is quite large from 34-285, but one thread accounts for 25-53\% of CPU utilization during
the period. All the other non-dominant threads consume a very small portion of the CPU utilization.

In common mobile applications, some levels of TLP are available. However, the thread composition is
quite different from the traditional multi-threaded applications with many homogeneous threads.
A main dominant thread accounts for a significant utilization and many non-dominant threads use
CPUs only for a short period of time. 

\begin{comment}
% Not Used
\begin{figure}[bt]
\begin{center}
\epsfig{figure=graphs/angry_thread.eps, width=8cm}
%\vspace{-0.45in}
\epsfig{figure=graphs/browser_thread.eps, width=8cm}
%\vspace{0.01in}
\end{center}
\caption{Application threads with 2 cores}
\label{fig:threads}
\end{figure}
\end{comment}

\begin{comment}
Figure~\ref{fig:threads} presents the CPU utilization of active threads with 2 core runs.
For both applications, the top thread dominates core utilization. For the game application,
there is only a small amount of utilization by the rest of threads, causing unbalanced loads
in two cores. For the browser application, there are several other active threads, which
can share the second core.
\end{comment}
