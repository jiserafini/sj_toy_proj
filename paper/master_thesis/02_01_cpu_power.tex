\chapter{The Power Behaviors of Mobile Platforms}
%\section{The Power Behaviors of Mobile Platforms}

In this section, we first measure and analyze the power consumption characteristics of mobile platforms with
various frequencies and utilization. We use the measured power data to construct a utility-based power model.
Secondly, we discuss the current power management techniques based on CPU utilization.

\section{CPU Power Characteristics}

To analyze the power consumption of multi-core smartphones, we use a state-of-a-art smart phone - Galaxy S3 - as our target device.
It has the Exynos 4412 CPU which includes four ARM Cortex-A9 cores, and runs Android 4.0.4 with Android kernel 3.0.15.
Exynos 4412 has 13 frequency levels from 200MHz to 1.4GHz and the detailed frequency and its voltage levels are shown in
Figure~\ref{fig:voltage}. Note that the voltage does not increase linearly with the frequency increase.
To measure the power consumption, Monsoon power meter is used, and the measured power is 
the whole system power.

\begin{figure}[bt]
\begin{center}
\epsfig{figure=graphs/freq_voltage.eps, width=7cm}
\vspace{-0.2in}
\end{center}
\caption{Voltages for different frequencies, normalized to the voltage at 200MHz}
\label{fig:voltage}
\end{figure}

\begin{figure}[bt]
\begin{center}
\epsfig{figure=graphs/power_util_consumption.eps, width=8cm}
\vspace{-0.1in}
\end{center}
\caption{Power consumption with various CPU utilization and frequency settings}
\label{fig:util_power}
\end{figure}

In this section, we first show how the core clock frequency and utilization affect the actual power
consumption of a quad-core processor used in the target smartphone. 
We use a micro-benchmark, which continuously uses CPU resources extensively to fully utilize CPUs and has high instruction
throughput per cycle. The benchmark can adjust CPU utilization by pausing the execution periodically. 
For this analysis, we vary three factors. Firstly, core clock frequencies are varied with DVFS from
the minimum 200MHz to the maximum 1.4GHz. Secondly, we vary the utilization of CPUs by forcing
the micro-benchmarks to pause periodically. The interactive
applications for smartphones commonly exhibit less than 100\% utilization for CPUs with 
a sequence of a short execution and idle period. 
We show how varying utilization changes the effect of DVFS on the system power.
Finally, we vary the number of active cores in the system.

Figure~\ref{fig:util_power} compares the measured power with different frequencies and utilizations with
one and four active cores. For the one-core configuration, the other three cores are turned off.
Firstly, as the utilization increases, the power increase is much higher
with higher clock frequencies than with lower clock frequencies. The system power is much more sensitive to
utilization when the clock frequency of cores is high.
Secondly, for each frequency level, the power increases almost linearly with the increase of utilization.
We will construct of a simple linear utility-based power model, which we will use to select an optimal
number of active cores for power management.

Thirdly, with low clock frequencies,
multiple cores do not consume significantly higher power than a single core with the same frequency, even under high utilization.
However, as the clock frequency increases, four cores consume almost four times power from a single core with
the same frequency. Confirming the common understanding of the energy efficiency of exploiting parallelism, 
using multiple low frequency cores provides better energy efficiency than a single core with high frequency. 
Figure~\ref{fig:1cpu_vs_2cpu} emphasizes the benefit of using low frequency multi-cores than
a single core with a high frequency. 

\begin{figure}[bt]
\begin{center}
\epsfig{figure=graphs/iso_perform_power.eps, width=7cm}
%\vspace{-0.2in}
\end{center}
\caption{iso-performance power consumption varying CPU setting}
\vspace{-0.1in}
\label{fig:1cpu_vs_2cpu}
\end{figure}

In Figure~\ref{fig:1cpu_vs_2cpu}, each section shows three different core and frequency settings
for the same performance goal. In this comparison, we assume an ideal parallelism, where
a task can be divided into multiple cores with perfect scaling. Furthermore, we also assume
that the application performance is proportional to the clock frequency.
For the three different performance goal, using as many cores with the lowest possible
frequency as possible, provide the best energy efficiency. The frequency should be 
lowered to set the utilization to 100\%. For the first performance goal,
using two cores at 1.2GHz consumes 50\% more power than using four cores at 600MHz with
100\% utilization 
However, such a power management for spreading computation to multiple cores as evenly as possible, depends
on the available TLP and also how each thread uses CPU resources. 

A strategy of using cores and DVFS can be inferred from the aforementioned analysis.
Firstly, the clock frequency must be lowered to the point where the utilization is close to 100\%.
If a balanced TLP is available, using two cores with a half clock frequency consume less power than
a single core. However, if TLP is unbalanced, using both cores may not always provide energy
efficiency, since the current mobile multi-cores often do not support per-core DVFS. Therefore,
the clock frequency must be adjusted, and cores can be turned on or off, so that
the utilization in both cores is closed to equally 100\%.

\begin{table}[tb]
\begin{center}
\begin{footnotesize}
\begin{tabular}{l|l|r}
\hline \hline
\multicolumn{2}{c|}{Mode}	& 	latency(msec)	\\
\hline
\multirow{2}{*}{Hotplugging}	&	On	&	17.251 \\\cline{2-3}
				&	Off	& 4.346 \\
\hline
\multicolumn{2}{c|}{DVFS}	&	1.045	\\
\hline
\multirow{2}{*}{Idle}		&	Clock gating	&	0.001 \\ \cline{2-3}
				&	HW power down	& 	0.300 \\
\hline \hline
\end{tabular}
\end{footnotesize}
\end{center}
%\vspace*{-0.15in}
\caption{Latency of Exynos 4412 AP}
% %\vspace*{-0.2in}
\end{table}

\begin{comment}
[Clock gating(WFI(Wait For Interrupt))]

WFI and WFE Standby modes disable most of the clocks of a processor, while keeping
its logic powered up. This reduces the power drawn to the static leakage current, leaving
a tiny clock power overhead requirement to enable the device to wake up.
Entry into WFI Standby mode is performed by executing the WFI instruction.


[ARM Power down mode]

Shutdown mode powers down the entire device, and all state, including cache, must be saved
externally by software. This state saving is performed with interrupts disabled, and finishes with
a Data Synchronization Barrier operation. The Cortex-A9 processor then communicates with a
power controller that the device is ready to be powered down in the same manner as when
entering Dormant Mode. The processor is returned to the run state by asserting reset.
\end{comment}
