\section{Power Behaviors of Multi-cores}

\subsection{Methodology}

%Odroid-A4 target & Galaxy Nexus(�ڸ������� ��)
%CPU: Exynos4210 Dual-core ARM Cortex-A9
%Android: 4.0.3
%Kernel: 3.0.15
%Power Meter: Monsoon Power meter (http://www.msoon.com/LabEquipment/PowerMonitor/)
To analyze the power consumption of multi-core smartphones, we use Odroid-A4 as our target device.
Its CPU is  Exynos4210 which includes two ARM Cortex-A9 cores, and
it runs Android 4.0.3 on Android kernel 3.0.15. To measure the power consumption, Monsoon power meter is used.
The measured power is the whole system power.

\subsection{Frequency and Utilization}

\begin{figure}[bt]
\begin{center}
\epsfig{figure=graphs/util_power.eps, width=7cm}
%\vspace{-0.2in}
\end{center}
\caption{Power consumption varying CPU utilization and frequency with CPU-stress}
\label{fig:util_power}
\end{figure}

\begin{figure}[bt]
\begin{center}
\epsfig{figure=graphs/1cpu_vs_2cpu.eps, width=7cm}
%\vspace{-0.2in}
\end{center}
\caption{Power consumption comparison: 1 vs 2 cores}
%\vspace{-0.2in}
\label{fig:1cpu_vs_2cpu}
\end{figure}

In this section, we first show how the core clock frequency and utilization affect the actual power
consumption of a dual-core processor used in the target smartphone. To stress the CPUs and memory system
differently, we use two micro-benchmarks, {\tt CPU-stress}, and {\tt mem-stress}. {\tt CPU-stress} 
continuously uses CPU resources extensively to fully utilize CPUs and has high instruction
throughput per cycle. {\tt Mem-stress} causes cache misses intentionally to incur all memory accesses
to be either L1 or L2 misses, depending on the working-set size. The CPU-stress and mem-stress represent two
extreme cases of using processors, and thus normal applications will behave within the range between 
the two cases.

For this analysis, we vary two factors. Firstly, core clock frequencies are varied with DVFS from
the minimum 200MHz to the maximum 1.2GHz. Secondly, we vary the utilization of CPUs by forcing
the micro-benchmarks to pause periodically to control the CPU utilization. The interactive
applications for smartphones commonly exhibit less than 100\% utilization for CPUs. We show how
varying utilization changes the effect of DVFS on the system power.

Figure~\ref{fig:util_power} compares the measured power with different frequencies and utilization
with {\tt CPU-stress}. The figures show the power consumptions with 1 and 2 cores. 
FIrstly, as the utilization increases, the power increase is much higher
with high clock frequencies than with low clock frequencies. The system power is much more sensitive to
utilization when the clock frequency of cores is high.
Secondly, with low clock frequencies,
two cores do not consume significantly higher power than a single core with the same frequency, even under high utilization.
However, as the clock frequency increases, two cores consume almost twice power from a single core with
the same frequency.

Figure~\ref{fig:1cpu_vs_2cpu} emphasizes the benefit of using low frequency dual-cores than 
a single core with a high frequency. In this figure, the power of three configurations are 
shown, 1 core with 1GHz, 1 core with 800MHz, and 2 cores with 500MHz. Assuming the perfect
parallelism, 2 cores with 500MHz may provide a similar performance to that with 1 core with
1GHz. However, the power consumption with two 500MHz cores is 20\% lower than those with one core with 1GHz.
Two 500MHz cores with 100\% utilization consume a similar power to a 800MHz core. 

\begin{figure}[bt]
\begin{center}
\epsfig{figure=graphs/util_power_stream.eps, width=7cm}
%\vspace{-0.25in}
\end{center}
\caption{Power consumption varying working set size and frequency with mem-stress}
%\vspace{-0.2in}
\label{fig:util_power_stream}
\end{figure}

Figure~\ref{fig:util_power_stream} compares {\tt mem-stress} with {\tt CPU-stress}.
In general, as the working set increases, and thus cache misses increase, less power is consumed.
Beyond 1M working set size, most of memory instructions access the external DRAM, reducing
the activity of cores.

A strategy of using cores and DVFS can be inferred from the aforementioned analysis.
Firstly, the clock frequency must be lowered to the point where the utilization is close to 100\%. 
If a balanced TLP is available, using two cores with a half clock frequency consume less power than
a single core. However, if TLP is unbalanced, using both cores may not always provide energy 
efficiency, since the current mobile multi-cores do not support per-core DVFS. Therefore, 
the clock frequency must be adjusted, and cores can be turned on or off, so that 
the utilization in both cores is closed to equally 100\%.

