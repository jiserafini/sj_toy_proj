%%
%% 감사의 글 시작
%% Acknowledgement
%%
% @command acknowledgement 감사의글
% @options [default: 클래스 옵션 korean|english ]
% - korean : 한글타이틀 | english : 영문타이틀

\acknowledgement[korean]


많은 분들의 도움으로 이 논문을 마무리할 수 있었습니다. 
가장 먼저 부족한 저에게 항상 훌륭한 가르침과 따뜻한 격려를 해주신 허재혁 교수님에게 감사 인사를 드립니다. 
2년 동안 교수님을 통하여 학술적 지식을 포함하여 많은 부분을 배울 수 있었습니다. 
또한, 논문 심사로 귀한 시간을 내주신 맹승렬 교수님, 김동준 교수님께도 감사의 말씀을 전합니다. 


이 논문은 컴퓨터 구조 연구실의 많은 분들의 도움이 있었기에 완성될 수 있었습니다. 
특히 제가 쓴 모든 논문에 공동 저자로 등재된 서원익과 김창대에게 고마움을 전합니다. 
원익이의 날카로운 조언과 창대의 폭넓은 지식을 통하여 이 논문의 내용이 더욱 견고해질 수 있었습니다. 
또한 아무 때고 연락해서 조언을 구해도 항상 밝은 모습으로 대답해주었던 김환주에게도 고마움을 전합니다. 

나이 많은 연구실 석사 신입생을 따뜻하게 받아주신 연구실 선배 김상훈 형, 설진호 형, 서본근 형, 박은지 선배, 영식, 대우, 한준, 효택, 진규, 성욱에게 이 자리를 빌려 감사 인사 드립니다. 
또한 항상 피자 먹으러 갈 때 멤버로 함께한 재호, 대훈, 정상, 나이 많은 신입 때문에 랩 장으로 고생 많았던 정호,
문화부로 즐거운 연구실 분위기를 이끌었던 정훈, 종종 물어보는 질문에도 웃으며 대답해주던 정섭과 재웅, 
형과 같은 조로 수업 듣느라 고생 많았던, 묵묵히 새벽 연구실을 지키던 상훈, 
육아와 회사일로 같이 고민을 함께하였던 동혁, 광원 형, 바쁜 회사일로 자주 보지는 못하지만 온라인 상으로 많은 격려를 주었던 서상원 선배, 재홍, 모두 제가 학교 생활을 하고 연구를 하는데 많은 도움을 주셨습니다. 
또한 연구실에서 다양한 경험을 함께 한 후배 건호, 진혁, 태훈과 같은 회사에서 진학하여 고민을 함께 나누었던 형호, 
연구실의 동기로서 졸업하고 새로운 출발하는 재원과 박사 과정에 진학하여 좀 더 깊이 있는 연구를 시작하는 보경에게도 고맙다는 말을 전합니다.

부족한 저에게 학술 연수를 통하여 더욱 발전할 수 있는 기회를 제공한 삼성전자, 특히 회사에서 지도 임원으로 과정에 충실하도록 많은 신경을 써주신 서상범 상무님, 학술 연수 전에 준비 과정에서 많은 도움을 주신 정석재 수석님, 
정진민 수석님, 박영주 책임님, 최진희 책임님, 논문을 준비하는 과정에서 많은 조언을 해주신 함명주 책임님께도 감사의 마음을 전합니다. 

철 없는 아들 때문에 걱정 많으신 아버지, 어머님과 귀한 딸을 지방에 내려 보내고 걱정이 많으셨을 장인어른, 장모님께도 이 자리를 빌려 앞으로 더욱 잘하겠다고 말씀 드립니다. 
마지막으로 신랑 따라 연고 하나 없는 대전에 100일도 안된 애기와 같이 무작정 내려와 2년간 재미있게 잘 지내준 아내 민영과 힘들 때마다 제 삶의 활력소가 되는 딸 서율에게도 항상 사랑하고, 고맙다고 전합니다. 
