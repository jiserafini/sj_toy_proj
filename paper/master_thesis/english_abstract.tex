\begin{abstract}

As smartphones proliferate, understanding 
the power behaviors of smartphones and their applications 
has become critical to overcome their battery limitation. However, 
the applications running on the smartphones
are commonly sensitive to user interaction, and their behaviors are 
different from traditional CPU benchmark applications. 
This paper characterizes the thread-level parallelism (TLP) and CPU usage patterns
of interactive smartphone applications. The paper presents how applications
exhibit different types of user interactions and how differently various interaction 
phases provide TLP. 

Based on the analysis on the TLP and interaction phases of mobile applications, 
we propose an improved power management scheme using the dynamic voltage and frequency scaling mechanism
available in mobile platforms. The proposed improvement uses a scheduling scheme,
which identifies and prioritizes a dominant thread, 
which determines the perceived performance of mobile applications.
Using a utility-based power model, the scheme finds the best number of active cores for
energy efficiency, and either pack or unpack non-dominant threads to the most 
energy efficient number of active cores.
In addition to the TLP-oriented scheduling component, the scheme also controls the headroom for
frequency scaling dynamically, to adapt to the user interaction phases. 
While maintaining the unnecessary headroom to the minimum, 
the proposed scheme improves the energy efficiency without any perceived performance
degradation.

\end{abstract}
