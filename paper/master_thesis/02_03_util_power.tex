\section{Utility-based Power Management}

In smartphone applications, cores are commonly underutilized, and the basic scheme for 
power management is to reduce the clock frequency to maximize the CPU utilization.
However, if the clock frequency is set to too low a frequency, the application can be slowed down
to the level perceivable to the user. One of the most commonly used utility-based
power management scheme is {\tt on-demand governor}. In this paper, our baseline
power management system uses the on-demand governor.

{\bf On-demand Governor: } 
Although various OSes may have various of policies to control the CPUs using DVFS mechanisms 
and one of the most widely used policies in mobile environment is {\tt on-demand} governor. 
The on-demand governor changes CPU frequency in response to its load at run time.
Performance can be slowed down if the load changes frequently, especially when the CPU load increases at a fast pace. When the CPU load increase suddenly, it may take several sampling periods
to match the frequency to the increased CPU load.
To avoid such a slow response to sudden load increases, the on-demand governor sets the highest available frequency 
when the load exceeds the up-threshold value to meet the user's expectation of the performance.
However, if the load falls below the down-threshold value, the frequency is decreased in steps.
Other widely used governors such as {\tt interactive} developed by Google and {\tt PegasusQ} developed by Samsung
are also based on the on-demand governor.
Algorithm~\ref{algo:ondemand} shows the sketch of the policy. If the utilization is higher than {\tt up-threshold},
the frequency is increased to the max frequency to improve the system responsiveness.

There are two limitations of the on-demand governor. Firstly, a conservative approach to preserve responsiveness 
may lead to unnecessary increase of energy consumption. It does not use any knowledge on the application
behaviors, to determine whether such a conservative policy is necessary for the applications.
Secondly,
the power governor works independently from thread scheduling, potentially losing opportunities for
further optimization. In this paper, based on the TLP and interaction phase analysis for mobile applications,
we will propose an improved power management scheme. 

\begin{algorithm}[t]
\caption{Ondemand Governor algorithm}
\label{algo:ondemand}
\begin{algorithmic}
{
%\FOR {every sampling rate}
\FOR {every core in the CPU} 
\STATE get $utilization$ since last check
\STATE
\STATE {\bf if} $utilization > UP\_THRESHOLD$
\STATE \quad increase frequency to MAX

\STATE
\STATE {\bf if} $utilization < DOWN\_THRESHOLD$
\STATE \quad {\bf then} decrease frequency to next low one
\ENDFOR
%\ENDFOR
}
\end{algorithmic}
\end{algorithm}

\begin{comment}
repeat every sampling rate
	for every core in the CPU
		get utilization since last check
		if (utilization > UP_THRESHOLD)
			increase frequency to MAX
		elif (utilization < DOWN_THRESHOLD)
			decrease frequency to next low one


\end{comment}
