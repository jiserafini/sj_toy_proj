% -*- TeX:UTF-8 -*-
%%
%% KAIST 학위논문양식 LaTeX용 (ver 0.4) 예시
%%
%% @version 0.4
%% @author  채승병 Chae,Seungbyung (mailto:chess@kaist.ac.kr)
%% @date    2004. 11. 12.
%%
%% @requirement
%% teTeX, fpTeX, teTeX 등의 LaTeX2e 배포판
%% + 은광희 님의 HLaTeX 0.991 이상 버젼 또는 홍석호 님의 HPACK 1.0
%% : 설치에 대한 자세한 정보는 http://www.ktug.or.kr을 참조바랍니다.
%%
%% @note
%% 기존에 널리 쓰여오던 차재춘 님의 학위논문양식 클래스 파일의 형식을
%% 따르지 않고 전면적으로 다시 작성하였습니다. 논문 정보 입력부분에서
%% 과거 양식과 다른 부분이 많으니 아래 예시에 맞춰 바꿔주십시오.
%% 
%%
%% @acknowledgement
%% 본 예시 논문은 물리학과 박사과정 김용현 님의 호의로 제공되었습니다.
%%
%% -------------------------------------------------------------------
%% @information
%% 이 예제 파일은 hangul-ucs를 사용합니다. UTF-8 입력 인코딩으로
%% 작성되었습니다. hlatex의 hfont는 이용하지 않습니다. --2006/02/11

% @class kaist.cls
% @options [default: doctor, korean, final]
% - doctor: 박사과정 | master : 석사과정
% - korean: 한글논문 | english: 영문논문
% - final : 최종판   | draft  : 시험판
% - pdfdoc : 선택하지 않으면 북마크와 colorlink를 만들지 않습니다.
\documentclass[master,english,final]{kaist-ucs}
% If you want make pdf document (include bookmark, colorlink)
%\documentclass[doctor,english,final,pdfdoc]{kaist-ucs}

% kaist.cls 에서는 기본으로 dhucs, ifpdf, graphicx 패키지가 로드됩니다.
% 추가로 필요한 패키지가 있다면 주석을 풀고 적어넣으십시오,
%\usepackage{...}
\usepackage{comment}
\usepackage{graphicx}
\usepackage{epsfig}
\usepackage{multirow}
\usepackage{algorithm}
\usepackage{algorithmic}


% @command title 논문 제목(title of thesis)
% @options [default: (none)]
% - korean: 한글제목(korean title) | english: 영문제목(english title)
\title[korean] {인터랙티브 모바일 어플리케이션 환경에서 TLP를 고려한 파워 관리 방안}
\title[english]{TLP-Aware Power Management for Interactive Mobile Applications}

% @note 표지에 출력되는 제목을 강제로 줄바꿈하려면 \linebreak 을 삽입.
%       \\ 나 \newline 등을 사용하면 안됩니다. (아래는 예시)
%
%\title[korean]{탄소 나노튜브의 물리적 특성에 대한\linebreak 이론 연구}
%\title[english]{Theoretical study on physical properties of\linebreak
%                carbon nanotubes}
%
% If you want to begin a new line in cover, use \linebreak .
% See examples above.
%


% @command author 저자 이름
% @param   family_name, given_name 성, 이름을 구분해서 입력
% @options [default: (none)]
% - korean: 한글이름 | chinese: 한문이름 | english: 영문이름
% 
% If you are a foreigner (this means you have no korean name),
% Write as follow
% \author[korean]{}{}
% \author[chinese]{family name in your native language}{given name in your native language}
% \author[english]{family name in english}{given name in english}
%

\author[korean] {우}{상 정}
\author[chinese]{禹}{相 禎}
\author[english]{Woo}{Sang-Jung}

% @command advisor 지도교수 이름 (복수가능)
% @usage   \advisor[options]{...한글이름...}{...영문이름...}{signed|nosign}
% @options [default: major]
% - major: 주 지도교수  | coopr: 공동 지도교수
\advisor[major]{허 재 혁}{Huh, Jaehyuk}{signed}	
%\advisor[coopr]{홍 길 동}{Gil-Dong Hong}{nosign}
%
% 지도교수 한글이름은 입력하지 않아도 됩니다.
% You may not input advisor's korean name
% like this \advisor[major]{}{Chang, Kee Joo}{signed}
%


% @command department {학과이름}{학위종류} - 아래 표에 따라 코드를 입력
% @command department {department code}{degree field} 
%
% department code table
%
% PH		// 물리학과 Department of Physics 
% MAS	// 수리과학과 Department of Mathematical Sciences
% CH 	// 화학과 Department of Chemistry
% NST	// 나노과학기술대학원 Graduate School of Nanoscience & Technology
% NT		// 나노과학기술 학제전공 Nano Science and Technology Program
% BS  	// 생명과학과 Department of Biological Sciences
% BIS	// 바이오및뇌공학과 Department of Bio and Brain Engineering
% MSE	// 의과학대학원 Graduate School of Medical Science and Engineering
% BM 	// 의과학 학제전공 Biomedical Science and Engineering Program
% CE 	// 건설및환경공학과 Department of Civil and Environmental Engineering
% ME 	// 기계공학전공 Division of Mechanical Engineering
% AE 	// 항공우주공학전공 Division of Aerospace Engineering
% OSE 	// 해양시스템공학전공 Department of Ocean Systems Engineering
% CBE 	// 생명화학공학과 Department of Chemical and Biomolecular Engineering
% MS 	// 신소재공학과 Department of Materials Science and Engineering
% NQE 	// 원자력 및 양자공학과 Department of Nuclear and Quantum Engineering
% EEW 	// EEWS 대학원 Graduate School of EEWS
% PSE 	// 고분자 학제전공 Polymer Science and Engineering Program
% SPE	// 우주탐사공학 학제전공 Space Exploration Engineering Program
% ENY 	// 환경・에너지공학 학제전공 Environmental and Energy Technology Program
% MSB 	// 경영과학과 Department of Management Science
% IT 		// 경영과학과(IT경영학) Department of Management Science (IT Business)
% BAP 	// 경영전문대학원프로그램 Master of Business Administration Program
% ITP 	// 글로벌IT기술대학원프로그램 Global Information & Telecommunication Technology Program
% ITM 	// 기술경영전문대학원 Graduate School of Innovation & Technology Management
% GCT 	// 문화기술대학원 Graduate School of Culture Technology
% CT 	// 문화기술(CT) 학제전공 Culture Technology Program
% EE 	// 전기 및 전자공학과 Department of Electrical Engineering 
% CS 	// 전산학과 Department of Computer Science 
% ICE 	// 정보통신공학과 Department of Information and Communications Engineering 
% IE 		// 산업및시스템공학과 Department of Industrial & Systems Engineering 
% KSE 	// 지식서비스공학과 Department of Knowledge Service Engineering 
% ID 		// 산업디자인학과 Department of Industrial Design 
% RE 	// 로봇공학 학제전공 Robotics Program
% STE 	// 반도체 학제전공 Semiconductor Technology Educational Program 
% SEP 	// 소프트웨어공학프로그램 Software Engineering Program 
% TE 	// 정보통신공학 학제전공 Telecommunication Engineering Program 
% EML 	// e-매뉴팩쳐링리더십 학제전공 e-Manufacturing Leadership Program
% MT 	// 경영공학과 Department of Management Engineering
% TM 	// 테크노경영전공 Techno-MBA Program
% FIN 	// 금융전문대학원 Graduate School of Finance and Accounting
%
% science: 이학 | engineering: 공학 | business : 경영학
% 박사논문의 경우는 학위종류를 입력하지 않아도 됩니다.
% If you write Ph.D. dissertation, you cannot input degree field.

\department{CS}{science}

% @command studentid 학번(ID)
\studentid{20113365}

% @command referee 심사위원 (석사과정 3인, 박사과정 5인)
\referee[1]{허 재 혁}
\referee[2]{맹 승 렬}
\referee[3]{김 동 준}
% \referee[5] {Barack Obama}
% Of course english name is available

% @command approvaldate 지도교수논문승인일
% @param   year,month,day 연,월,일 순으로 입력
\approvaldate{2002}{12}{5}

% @command refereedate 심사위원논문심사일
% @param   year,month,day 연,월,일 순으로 입력
\refereedate{2012}{12}{17}

% @command gradyear 졸업년도
\gradyear{2013}

%%%%%%%%%%%%%%%%%%%%%%%%%%%%%%%%%%%%%%%%%%%%%%%%%%%%%%%%%%%%%%%%%%%%%%%%%%%%%%%%%%%%%%%%%%%%%%%%%%%%%%%%%%%%%%%%%%%%%%%%%%%%

% 본문 시작
\begin{document}

    % 앞표지, 속표지, 학위논문 제출승인서, 학위논문 심사완료 검인서는
    % 클래스 옵션을 final로 지정해주면 자동으로 생성되며,
    % 반대로 옵션을 draft로 지정해주면 생성되지 않습니다.

    % 영문초록 (abstract)
	\begin{abstract}

As smartphones proliferate, understanding 
the power behaviors of smartphones and their applications 
has become critical to overcome their battery limitation. However, 
the applications running on the smartphones
are commonly sensitive to user interaction, and their behaviors are 
different from traditional CPU benchmark applications. 
This paper characterizes the thread-level parallelism (TLP) and CPU usage patterns
of interactive smartphone applications. The paper presents how applications
exhibit different types of user interactions and how differently various interaction 
phases provide TLP. 

Based on the analysis on the TLP and interaction phases of mobile applications, 
we propose an improved power management scheme using the dynamic voltage and frequency scaling mechanism
available in mobile platforms. The proposed improvement uses a scheduling scheme,
which identifies and prioritizes a dominant thread, 
which determines the perceived performance of mobile applications.
Using a utility-based power model, the scheme finds the best number of active cores for
energy efficiency, and either pack or unpack non-dominant threads to the most 
energy efficient number of active cores.
In addition to the TLP-oriented scheduling component, the scheme also controls the headroom for
frequency scaling dynamically, to adapt to the user interaction phases. 
While maintaining the unnecessary headroom to the minimum, 
the proposed scheme improves the energy efficiency without any perceived performance
degradation.

\end{abstract}


    % 목차 (Table of Contents) 생성
    \tableofcontents

    % 표목차 (List of Tables) 생성
    \listoftables

    % 그림목차 (List of Figures) 생성
    \listoffigures

    % 위의 세 종류의 목차는 한꺼번에 다음 명령으로 생성할 수도 있습니다.
    %\makecontents

%% 이하의 본문은 LaTeX 표준 클래스 report 양식에 준하여 작성하시면 됩니다.
%% 하지만 part는 사용하지 못하도록 제거하였으므로, chapter가 문서 내의
%% 최상위 분류 단위가 됩니다.
%% You cannot use 'part'


% Introduction	
\section{Introduction}

Although smartphones dominate mobile platforms, encroaching the turf of traditional computing platforms such
as laptop and desktop systems, the primary factor limiting their mobility is the energy efficiency and thus the battery 
life. Among the components in the smartphones, processors have been considered to be one of the top two power-hungry
components, along with the display device~\cite{smartphonepower,intothewild}. Meanwhile, the processors for smartphones have 
evolved into multi-cores, making the exploitation of thread-level parallelism (TLP) critical 
for the performance and energy efficiency~\cite{desktopTLP}. To fully utilize the potential benefit of multi-cores, and 
to improve the energy efficiency, it is critical to understand the behaviors of smartphone applications for their
available TLP.

However, such smartphone applications exhibit quite different behaviors from the traditional benchmark applications
used for servers and desktops. Their executions are highly dependent upon the interactions with users, and the
power consumption behaviors are also affected by user interactions. Although the importance of understanding such
interactive applications has been growing, there  have been a limited number of studies on the behavior of 
smartphone applications~\cite{intothewild, characterinteractive}.

This paper characterizes three different aspects of power behaviors in commercial smartphone platforms and 
their applications. Firstly, this study analyzes the actual power consumption of multi-cores used for
smartphones with varying clock frequencies and CPU utilization. We used two types of micro-benchmarks,
one for stressing CPUs and the other for stressing the memory system. The analysis shows how the 
current multi-cores and their support for dynamic voltage and frequency scaling (DVFS) consume power 
with the two extreme cases of application types.

Secondly, we characterize the parallelism and CPU utilization behaviors of smartphone applications.
Gutierrez et al. showed the characteristics of smartphone applications, focusing their micro-architectural
aspects~\cite{characterinteractive}. This paper focuses on the thread-level parallelism (TLP) and 
CPU utilization, which are directly related to the power management with DVFS and dynamic core plug-ins~\cite{packandcap}.
We use user interaction as a delimiting factor to identify phases in interactive applications.
Each application has different behaviors with user interactions through touch screen, and we show
how TLP and utilization change differently for various application phases.

Finally, we show how the state-of-art power management system in smartphones behaves with 
the interactive applications. Such a power management system available in the current smartphone operating system
attempts to minimize wasted cycles with DVFS. At the same time, the power management system must not affect the users' perception
on system performance and interactivity. However, since the current multi-cores for smartphones do not support per-core
DVFS due to the area limitation, the power management system may miss the opportunity to further 
reduce the power consumption. This paper explores where the further opportunities for energy reduction exist with
multi-core platforms and interactive applications.




% 2 Mobile Power Behaviors
\chapter{The Power Behaviors of Mobile Platforms}
%\section{The Power Behaviors of Mobile Platforms}

In this section, we first measure and analyze the power consumption characteristics of mobile platforms with
various frequencies and utilization. We use the measured power data to construct a utility-based power model.
Secondly, we discuss the current power management techniques based on CPU utilization.

\section{CPU Power Characteristics}

To analyze the power consumption of multi-core smartphones, we use a state-of-a-art smart phone - Galaxy S3 - as our target device.
It has the Exynos 4412 CPU which includes four ARM Cortex-A9 cores, and runs Android 4.0.4 with Android kernel 3.0.15.
Exynos 4412 has 13 frequency levels from 200MHz to 1.4GHz and the detailed frequency and its voltage levels are shown in
Figure~\ref{fig:voltage}. Note that the voltage does not increase linearly with the frequency increase.
To measure the power consumption, Monsoon power meter is used, and the measured power is 
the whole system power.

\begin{figure}[bt]
\begin{center}
\epsfig{figure=graphs/freq_voltage.eps, width=7cm}
\vspace{-0.2in}
\end{center}
\caption{Voltages for different frequencies, normalized to the voltage at 200MHz}
\label{fig:voltage}
\end{figure}

\begin{figure}[bt]
\begin{center}
\epsfig{figure=graphs/power_util_consumption.eps, width=8cm}
\vspace{-0.1in}
\end{center}
\caption{Power consumption with various CPU Util. and Freq. settings}
\label{fig:util_power}
\end{figure}

In this section, we first show how the core clock frequency and utilization affect the actual power
consumption of a quad-core processor used in the target smartphone. 
We use a micro-benchmark, which continuously uses CPU resources extensively to fully utilize CPUs and has high instruction
throughput per cycle. The benchmark can adjust CPU utilization by pausing the execution periodically. 
For this analysis, we vary three factors. Firstly, core clock frequencies are varied with DVFS from
the minimum 200MHz to the maximum 1.4GHz. Secondly, we vary the utilization of CPUs by forcing
the micro-benchmarks to pause periodically. The interactive
applications for smartphones commonly exhibit less than 100\% utilization for CPUs with 
a sequence of a short execution and idle period. 
We show how varying utilization changes the effect of DVFS on the system power.
Finally, we vary the number of active cores in the system.

Figure~\ref{fig:util_power} compares the measured power with different frequencies and utilizations with
one and four active cores. For the one-core configuration, the other three cores are turned off.
Firstly, as the utilization increases, the power increase is much higher
with higher clock frequencies than with lower clock frequencies. The system power is much more sensitive to
utilization when the clock frequency of cores is high.
Secondly, for each frequency level, the power increases almost linearly with the increase of utilization.
We will construct of a simple linear utility-based power model, which we will use to select an optimal
number of active cores for power management.

Thirdly, with low clock frequencies,
multiple cores do not consume significantly higher power than a single core with the same frequency, even under high utilization.
However, as the clock frequency increases, four cores consume almost four times power from a single core with
the same frequency. Confirming the common understanding of the energy efficiency of exploiting parallelism, 
using multiple low frequency cores provides better energy efficiency than a single core with high frequency. 
Figure~\ref{fig:1cpu_vs_2cpu} emphasizes the benefit of using low frequency multi-cores than
a single core with a high frequency. 

\begin{figure}[bt]
\begin{center}
\epsfig{figure=graphs/iso_perform_power.eps, width=7cm}
%\vspace{-0.2in}
\end{center}
\caption{Iso-performance power consumption varying CPU setting}
\vspace{-0.1in}
\label{fig:1cpu_vs_2cpu}
\end{figure}

In Figure~\ref{fig:1cpu_vs_2cpu}, each section shows three different core and frequency settings
for the same performance goal. In this comparison, we assume an ideal parallelism, where
a task can be divided into multiple cores with perfect scaling. Furthermore, we also assume
that the application performance is proportional to the clock frequency.
For the three different performance goal, using as many cores with the lowest possible
frequency as possible, provide the best energy efficiency. The frequency should be 
lowered to set the utilization to 100\%. For the first performance goal,
using two cores at 1.2GHz consumes 50\% more power than using four cores at 600MHz with
100\% utilization 
However, such a power management for spreading computation to multiple cores as evenly as possible, depends
on the available TLP and also how each thread uses CPU resources. 

Modern mobile CPU supports various low power techniques such as CPU Hotplugging and DVFS.
However, the latency is different for each mode. 
Table~\ref{tab:exynos_latency} shows the different latency of each mode.
In all cases, CPU Hotplugging has long latency in comparison with DVFS, 
especially CPU Hotplugging-On is 17 times longer. 
In mobile system, long latency may degrade user perceived performance 
since the system cannot meet the applications' performance request. 
According to previous experimental result, 
DVFS is no significant power consumption difference in comparison with Hotplugging.

\begin{table}[tb]
\begin{center}
\begin{footnotesize}
\begin{tabular}{l|l|r}
\hline \hline
\multicolumn{2}{c|}{Mode}	& 	latency(msec)	\\
\hline
\multirow{2}{*}{Hotplugging}	&	On	&	17.251 \\\cline{2-3}
				&	Off	& 4.346 \\
\hline
\multicolumn{2}{c|}{DVFS}	&	1.045	\\
\hline
\multirow{2}{*}{Idle}		&	Clock gating	&	0.001 \\ \cline{2-3}
				&	HW power down	& 	0.300 \\
\hline \hline
\end{tabular}
\end{footnotesize}
\end{center}
%\vspace*{-0.15in}
\caption{Latency of Exynos 4412 AP}
% %\vspace*{-0.2in}
\end{table}

\begin{comment}
[Clock gating(WFI(Wait For Interrupt))]

WFI and WFE Standby modes disable most of the clocks of a processor, while keeping
its logic powered up. This reduces the power drawn to the static leakage current, leaving
a tiny clock power overhead requirement to enable the device to wake up.
Entry into WFI Standby mode is performed by executing the WFI instruction.


[ARM Power down mode]

Shutdown mode powers down the entire device, and all state, including cache, must be saved
externally by software. This state saving is performed with interrupts disabled, and finishes with
a Data Synchronization Barrier operation. The Cortex-A9 processor then communicates with a
power controller that the device is ready to be powered down in the same manner as when
entering Dormant Mode. The processor is returned to the run state by asserting reset.
\end{comment}


A strategy of using cores and DVFS can be inferred from the aforementioned analysis.
Firstly, the clock frequency must be lowered to the point where the utilization is close to 100\%.
If a balanced TLP is available, using two cores with a half clock frequency consume less power than
a single core. However, if TLP is unbalanced, using both cores may not always provide energy
efficiency, since the current mobile multi-cores often do not support per-core DVFS. Therefore,
the clock frequency must be adjusted, and cores can be turned on or off, so that
the utilization in both cores is closed to equally 100\%.



\section{CPU Power Model}

Based on the linear power increase with utilization at the same frequency,
we construct a simple utility-based power model. A similar model
has been used in several other prior studies~\cite{VMpower, isca11DVFS}.
The power consumption of CPU is composed of the static and dynamic power,
and also the power can be consumed either by the shared components and processor cores.

The following equation simply represents the static and dynamic
power consumed by the shared components and multiple cores.

\begin{center}
\vspace{-0.05in}
$Power_{CPU} = S_{shared} + D_{shared} + \sum_{i=i}^{N} (S_{core_i} + D_{core_i})$
\vspace{-0.05in}
\end{center}

$S$ and $D$ denote the static and dynamic power respectively, and $N$ is the number of power-supplied cores. 
Note that a core can be powered-off by the hot-pluging technique.

The dynamic power of each core is proportional to the core utilization,
and the dynamic power of the shared component is proportional to the sum
of utilizations of all the active cores. 
Since the static power of each core should be equal ($S_{core}$),
the power consumption can be modeled as the following equation.
In the equation, the dynamic power of the shared components is
merged with the dynamic power of cores.

\begin{center}
\vspace{-0.05in}
$Power_{CPU} = S_{shared} + N * S_{core} + \sum_{i=i}^{N} \alpha_i * Utilization_i$
\vspace{-0.05in}
\end{center}

In this equation, there are three constants: $S_{shared}$, $S_{core}$, and $\alpha_i$.
As shown in Figure~\ref{fig:util_power}, $\alpha_i$ is directly related to the core frequency.
If we have the constants for each supported frequency and the CPU utilization,
we can construct a utility-based estimation model for the CPU power consumption.
Using the results from figure~\ref{fig:util_power}, we obtain the constants for each frequency.
To validate the model accuriacy, the dotted line of figure~\ref{fig:util_power} shows the 
estimated power consumption.
The average error ratio is 0.27\% and the trends are successfully estimated.


\section{Utility-based Power Management}

In smartphone applications, cores are commonly underutilized, and the basic scheme for 
power management is to reduce the clock frequency to maximize the CPU utilization.
However, if the clock frequency is set to too low a frequency, the application can be slowed down
to the level perceivable to the user. One of the most commonly used utility-based
power management scheme is {\tt on-demand governor}. In this paper, our baseline
power management system uses the on-demand governor.

{\bf On-demand Governor: } 
Although various OSes may have various of policies to control the CPUs using DVFS mechanisms 
and one of the most widely used policies in mobile environment is {\tt on-demand} governor. 
The on-demand governor changes CPU frequency in response to its load at run time.
Performance can be slowed down if the load changes frequently, especially when the CPU load increases at a fast pace. When the CPU load increase suddenly, it may take several sampling periods
to match the frequency to the increased CPU load.
To avoid such a slow response to sudden load increases, the on-demand governor sets the highest available frequency 
when the load exceeds the up-threshold value to meet the user's expectation of the performance.
However, if the load falls below the down-threshold value, the frequency is decreased in steps.
Other widely used governors such as {\tt interactive} developed by Google and {\tt PegasusQ} developed by Samsung
are also based on the on-demand governor.
Algorithm~\ref{algo:ondemand} shows the sketch of the policy. If the utilization is higher than {\tt up-threshold},
the frequency is increased to the max frequency to improve the system responsiveness.

There are two limitations of the on-demand governor. Firstly, a conservative approach to preserve responsiveness 
may lead to unnecessary increase of energy consumption. It does not use any knowledge on the application
behaviors, to determine whether such a conservative policy is necessary for the applications.
Secondly,
the power governor works independently from thread scheduling, potentially losing opportunities for
further optimization. In this paper, based on the TLP and interaction phase analysis for mobile applications,
we will propose an improved power management scheme. 

\begin{algorithm}[t]
\caption{Ondemand Governor algorithm}
\label{algo:ondemand}
\begin{algorithmic}
{
%\FOR {every sampling rate}
\FOR {every core in the CPU} 
\STATE get $utilization$ since last check
\STATE
\STATE {\bf if} $utilization > UP\_THRESHOLD$
\STATE \quad increase frequency to MAX

\STATE
\STATE {\bf if} $utilization < DOWN\_THRESHOLD$
\STATE \quad {\bf then} decrease frequency to next low one
\ENDFOR
%\ENDFOR
}
\end{algorithmic}
\end{algorithm}

\begin{comment}
repeat every sampling rate
	for every core in the CPU
		get utilization since last check
		if (utilization > UP_THRESHOLD)
			increase frequency to MAX
		elif (utilization < DOWN_THRESHOLD)
			decrease frequency to next low one


\end{comment}


% 3. TLP Behaviros of Mobile Applications
\chapter{TLP Behavior of Mobile Applications}

\section{Phase Classification}

In mobile applications, the user interactions dominate the CPU usage patterns, and
affect the TLP and utilization significantly. Although each application exhibits
different types of user interactions and CPU usage patterns, we broadly classify
the execution phases of mobile applications into three phases, {\tt setup},
{\tt responsive}, and {\tt interactive}.
Firstly, a {\tt setup} phase initializes an application, or loads data for a new round 
of game. In this period, user interactions do not affect the CPU usage significantly.
Secondly, a {\tt responsive} phase occurs when cores become significantly active only after a user interaction.
In such a responsive phase, CPUs become idle or show low utilization sometime after the interaction, after
the computation is completed to process the user interaction. A common example of
the responsive phase is the interaction with a web browser application. Once
a hyperlink is touched, the CPUs become active to render a new page. However,
once the process is completed, the CPUs become idle again.

Finally, in an {\tt interactive} phase, the application continuously conducts some work
while users may interact with the application. For games, 
CPUs process the game engine continuously regardless of user interactions. At the same time,
user interactions may change the behaviors of such computational work.
However, even in an interactive phase, the CPU is not completely occupied with 
active threads, and the CPU utilization do not change significantly.
An application may exhibit only a subset of the three phases. For example, some applications
may not have any responsive phases.

\begin{figure}[bt]
\begin{center}
\epsfig{figure=graphs/angrybird_interactive.eps, width=8cm}
%\vspace{-0.45in}
\epsfig{figure=graphs/browser_1cpu.eps, width=8cm}
%\vspace{0.01in}
\end{center}
\caption{CPU util. and touch events with 1 core}
%\vspace{-0.2in}
\label{fig:interact_util_1cpu}
\end{figure}

To show the detailed behaviors of applications, Figure~\ref{fig:interact_util_1cpu} presents
the utilization of a single 1GHz core for a game application (Angry bird) and a web browser.
The dots represent user touch inputs on the touch screen.
For the game application, interactive phases dominate the execution.
In an interactive phase,
CPU utilization is maintained around 40\%, regardless of user interactions.
During the interactive phase, even if the user affects the application execution though
touch inputs, the CPU utilization is dominantly determined by the game engine routine.
However, in the browser application, {\tt response} phases are mostly
shown. CPU utilization increases only when a user touch occurs, and the utilization
drop soon after the interaction is processed.

\begin{table}[tb]
\begin{center}
\begin{footnotesize}
\begin{tabular}{l|l|l}
\hline \hline
Main Phase & App & Description \\
\hline
\multirow{4}*{Interactive}	& Trial Winter		& 3D motorcycle racing game	\\ \cline{2-3}
& Eternity Warriors 	& 3D action RPG game		\\\cline{2-3}
& Angry Brid 		& Shooting game with physics engine \\\cline{2-3}
& Movie Player		& Play a video file \\\cline{2-3}
\hline
\multirow{5}*{Responsive}	&Photo Editor		& Apply various filter to the photo \\\cline{2-3}
& Google Map			& Search a place and magnify \\\cline{2-3}
& ezPDF Reader		& Read the pdf document \\\cline{2-3}
& Browser				& BBench on default browser \\\cline{2-3}
& Mobile Chrome		& BBench on Chrome browser \\
\hline \hline
\end{tabular}
\end{footnotesize}
\end{center}
%\vspace*{-0.15in}
\caption{Test Application Set}
%\vspace*{-0.2in}
\label{tab:app_summary}
\end{table}


Although applications have all or a subset of the three phases, one of the three phases
dominate each application. For example, for the majority of game applications, 
the interactive phase appears frequently. Web browsers are commonly dominated by
responsive phases. Table~\ref{tab:app_summary} shows a set of benchmark smartphone 
applications classified by their main dominant phase.
The first four applications are games and a movie player, and the applications
have all three phases. However, for the four applications, interactive phases
dominate the execution time of CPUs.
In the next five applications, responsive phases dominate application execution. In those applications,
interactive phases do not appear, since the most of CPU usage is initiated by user inputs.

\begin{table}[tb]
\begin{center}
\begin{footnotesize}
\begin{tabular}{l|l|r|r}
\hline \hline
App		& Metric	& w/o touch		& w/ touch \\
\hline
\multirow{2}*{Trial Winter}	& CPU util.(\%)	& 68.93	& 76.39	\\
							& \# active threads	& 1.36	& 1.34	\\
\hline
\multirow{2}*{Eternity Warriors} & CPU util.(\%)	& 87.44	& 93.19	\\
							& \# active threads	& 1.49	& 1.88	\\
\hline
\multirow{2}*{Angry Brid}	&  CPU util.(\%)	& 41.92	& 48.14	\\
					& \# active threads & 0.64	& 1.43	\\
\hline
\multirow{2}*{Movie Player}		&  CPU util.(\%)	& 22.78	& 25.21	\\
					& \# active threads	& 0.20	& 0.83 \\
\hline \hline
\multirow{2}*{Photo Editor}		&  CPU util.(\%)	& 2.22	 & 73.42	\\
					& \# active threads	 & 0.03	& 3.42	\\
\hline
\multirow{2}*{Google Map}		& CPU util.(\%)	& 23.12& 86.69 \\
					& \# active threads	& 0.46	& 2.06	\\
\hline
\multirow{2}*{ezPDF Reader}	& CPU util.(\%)	& 1.45	& 43.71	\\
					& \# active threads & 0.03	& 1.04 \\
\hline
\multirow{2}*{Browser} 	& CPU util.(\%)	& 12.23	& 86.92	\\
					& \# active threads & 0.22	& 4.21	\\
\hline
\multirow{2}*{Mobile Chrome}	& CPU util.(\%)	& 0.58	& 95.05	\\
					& \# active threads & 0.03	& 2.43	\\
\hline \hline
\end{tabular}
\end{footnotesize}
\end{center}
%\vspace*{-0.15in}
\caption{Summary of Application Phase Characteristics (One core 1.4GHz)}
%\vspace*{-0.2in}
\label{tab:tlp_summary}
\end{table}


\section {Available TLP}

In this section, we present the CPU utilization and available TLP during the main phase
of each type of applications. 
Table~\ref{tab:tlp_summary} presents the utilization and TLP for our benchmark applications in
their main phase, the first four with the interactive phase, and the rest with
the responsive phase.

For each phase of an application, two numbers are shown. Firstly,
CPU utilization is an average utilization for one core with 1.4Ghz during the phase. 
The second row, the number of
active threads presents the available parallelism. 
A phase is further decomposed into a portion of the phase with and without
user interactions. 

Firstly, the overall utilization is relatively low for mobile applications. The utilization is
only measured with one core with 1.4Ghz. Considering four available cores in the target platform,
even a single core is not fully utilized. For games, the CPU utilization is between 48-93\%
with a core, but the movie player exhibits much lower 25\% utilization. For interactive
applications, touch inputs do not affect the CPU utilization significantly. However, as expected,
the utilizations for responsive applications are high only with touch inputs. 

Secondly, the average number of active threads during the main phase is quite limited. Most
of applications have only 1-2 active threads at any given time to be scheduled to cores. 
An exception is the web browser which has more than four active threads. For such applications
with several threads, multiple cores can be used with much lower clock frequencies to reduce
the overall energy.

\section{Dominant Thread}

\begin{table}[tb]
\begin{center}
\begin{footnotesize}
\begin{tabular}{l|l|r|r}
\hline \hline
App		& Metric	& w/o touch		& w/ touch \\
\hline
\multirow{2}*{Trial Winter}	& Avg. portion(\%)	& 44.67	& 44.92	\\
						& \# of exec thread & 55 & 55 \\
\hline						
\multirow{2}*{Eternity Warriors}	& Avg. portion(\%)	& 33.65	& 37.73	\\
					& \# of exec thread & 103 & 107 \\
\hline
\multirow{2}*{Angry Brid}			& Avg. portion(\%)	& 35.14	& 37.46	\\
					& \# of exec thread & 50 & 53 \\
\hline
\multirow{2}*{Movie Player}	& Avg. portion(\%)	& \multicolumn{1}{c|}{-}	& \multicolumn{1}{c}{-}	\\
					& \# of exec thread & 30  & 34 \\
\hline	\hline						
\multirow{2}*{Photo Editor}	& Avg. portion(\%)	&\multicolumn{1}{c|}{-}	& 53.42\\
					& \# of exec thread &  \multicolumn{1}{c|}{-} & 138 \\
\hline
\multirow{2}*{Google Map}	& Avg. portion(\%)	& \multicolumn{1}{c|}{-}	& 25.55	\\
					& \# of exec thread & \multicolumn{1}{c|}{-} & 275 \\
\hline
\multirow{2}*{ezPDF Reader}		& Avg. portion(\%)	& \multicolumn{1}{c|}{-}	& 34.96	\\
					& \# of exec thread & \multicolumn{1}{c|}{-} & 65 \\
\hline
\multirow{2}*{Browser}	& Avg. portion(\%)	& \multicolumn{1}{c|}{-}	& 29.12	\\
					& \# of exec thread & \multicolumn{1}{c|}{-} 	&  285 \\
\hline
\multirow{2}*{Mobile Chrome}	& Avg. portion(\%)	& \multicolumn{1}{c|}{-}	& 38.16	\\
					& \# of exec thread & \multicolumn{1}{c|}{-}	& 89\\
\hline	\hline				
\end{tabular}
\end{footnotesize}
\end{center}
%\vspace*{-0.15in}
\caption{Analysis of Dominant Thread (four 1GHz cores)}
\label{tab:dom_threads}
%\vspace*{-0.2in}
\end{table}

\begin{comment}
\begin{table}[tb]
\begin{center}
\begin{footnotesize}
\begin{tabular}{l|l|r|r}
\hline \hline
App		& Metric	& w/o touch		& w/ touch \\
\hline
\multirow{3}*{Trial Winter}	& Avg. portion(\%)	& 44.67	& 44.92	\\
						& \# of exec thread & 55 & 55 \\
						& Avg. relation & 1.00	& 1.00	\\
\hline						
\multirow{3}*{Eternity Warriors}	& Avg. portion(\%)	& 33.65	& 37.73	\\
					& \# of exec thread & 103 & 107 \\
					& Avg. relation & 0.95	& 0.93	\\
\hline
\multirow{3}*{Angry Brid}			& Avg. portion(\%)	& 35.14	& 37.46	\\
					& \# of exec thread & 50 & 53 \\
					& Avg. relation	& 1.00	& 1.00	\\
\hline
\multirow{3}*{Movie Player}	& Avg. portion(\%)	& \multicolumn{1}{c|}{-}	& \multicolumn{1}{c}{-}	\\
					& \# of exec thread & 30  & 34 \\
					& Avg. relation 	& \multicolumn{1}{c|}{-}	& \multicolumn{1}{c}{-}	\\
\hline	\hline						
\multirow{3}*{Photo Editor}	& Avg. portion(\%)	&\multicolumn{1}{c|}{-}	& 53.42\\
					& \# of exec thread &  \multicolumn{1}{c|}{-} & 138 \\\
					& Avg. relation		&\multicolumn{1}{c|}{-}		& 0.67	\\
\hline
\multirow{3}*{Google Map}	& Avg. portion(\%)	& \multicolumn{1}{c|}{-}	& 25.55	\\
					& \# of exec thread & \multicolumn{1}{c|}{-} & 275 \\
					& Avg. relation 	& \multicolumn{1}{c|}{-}	& 0.59	\\
\hline
\multirow{3}*{ezPDF Reader}		& Avg. portion(\%)	& \multicolumn{1}{c|}{-}	& 34.96	\\
					& \# of exec thread & \multicolumn{1}{c|}{-} & 65 \\
					& Avg. relation 	& \multicolumn{1}{c|}{-}	& 0.66	\\
\hline
\multirow{3}*{Browser}	& Avg. portion(\%)	& \multicolumn{1}{c|}{-}	& 29.12	\\
					& \# of exec thread & \multicolumn{1}{c|}{-} 	&  285 \\
					& Avg. relation 	& \multicolumn{1}{c|}{-} 	& 0.56	\\
\hline
\multirow{3}*{Mobile Chrome}	& Avg. portion(\%)	& \multicolumn{1}{c|}{-}	& 38.16	\\
					& \# of exec thread & \multicolumn{1}{c|}{-}	& 89\\
					& Avg. relation 	& \multicolumn{1}{c|}{-}	& 0.72	\\
\hline	\hline				
\end{tabular}
\end{footnotesize}
\end{center}
%\vspace*{-0.15in}
\caption{Analysis of Dominant Thread (four 1GHz cores)}
\label{tab:dom_threads}
%\vspace*{-0.2in}
\end{table}
\end{comment}



\begin{figure}[relative_util]
\begin{center}
\epsfig{figure=graphs/relative_util_normalized_min_exec_time.eps, width=8cm}
%\vspace{-0.45in}
\end{center}
\caption{Relative Util. Normalized to Minimum Thread Exec. Time}
%\vspace{-0.2in}
\label{fig:relative_util}
\end{figure}



To minimize the power for multi-cores, more cores should be used with the lowest possible frequency.
However, it must be able to accommodate the active threads and their utilization requirements. 
Another important aspect of TLP is how each thread consumes CPU resources. Although there is 
a limited number of active threads in mobile applications during their main phases, the CPU utilizations
of the threads are quite skewed. One or two threads dominate the overall CPU utilization, even though
there are many other threads, which shortly use CPU.

Table~\ref{tab:dom_threads} presents the behaviors of dominant threads in our benchmark applications.
For each application, the first line shows the average portion of utilization of a dominant thread during
the main phase. For example, for angry bird, a single dominant thread uses 37\% of overall CPU utilization 
during the phase. The second line shows the number of threads, which are activated at least once 
during the main phase. The number of
executed threads is quite large from 34-285, but one thread accounts for 25-53\% of CPU utilization during
the period. All the other non-dominant threads consume a very small portion of the CPU utilization.

In common mobile applications, some levels of TLP are available. However, the thread composition is
quite different from the traditional multi-threaded applications with many homogeneous threads.
A main dominant thread accounts for a significant utilization and many non-dominant threads use
CPUs only for a short period of time. 

\begin{comment}
% Not Used
\begin{figure}[bt]
\begin{center}
\epsfig{figure=graphs/angry_thread.eps, width=8cm}
%\vspace{-0.45in}
\epsfig{figure=graphs/browser_thread.eps, width=8cm}
%\vspace{0.01in}
\end{center}
\caption{Application threads with 2 cores}
\label{fig:threads}
\end{figure}
\end{comment}

\begin{comment}
Figure~\ref{fig:threads} presents the CPU utilization of active threads with 2 core runs.
For both applications, the top thread dominates core utilization. For the game application,
there is only a small amount of utilization by the rest of threads, causing unbalanced loads
in two cores. For the browser application, there are several other active threads, which
can share the second core.
\end{comment}


% 4. TLP-aware/Phase-aware DVFS
\chapter{Improving Power Management for Mobile Platforms}

In this section, we propose three techniques to improve the energy efficiency of
mobile platforms, exploiting the TLP characteristics and interaction types of
applications. The first technique identifies a dominant thread and
assign a dedicated core to the dominant thread, to preserve the context
in the cache, and avoid any unnecessary interference with other threads.
The second technique attempts to maximize the utilization by lowering
the clock frequency as much as possible. To avoid any performance degradation,
it exploits the usage difference between interactive and responsive phases.
The third technique determines the most energy efficient number of
active cores for a given set of current active threads. Using a utility-based
power model, it picks the number of cores, and assign non-dominant threads
evenly to the cores.
Algorithm~\ref{algo:tlpdvfs} shows the proposed power management with the 
three techniques ({\tt part 2, part 3, and part 4} ).

\section{Dominant Thread Scheduling for Energy Efficiency}

The first technique identifies and prioritizes the execution of
a dominant thread. As shown in Section 3.3, 
common mobile applications have a dominant thread, and often,
the performance perceived by the user is determined by the delay of 
the dominant thread. Not to degrade the user perceived performance,
any power management scheme must not increase the delay of
the dominant thread. Furthermore, reducing the delay of a dominant thread
can potentially reduce the CPU energy by completing a phase as early
as possible.

In the proposed power management scheme, a dominant thread is identified
with the energy management module, tracking the CPU utilization of 
threads dynamically. Once a dominant thread is identified, the thread
is scheduled to a fixed core. Assigning a fixed core to a dominant thread
has several benefits. Any interference, including the cache pollution, 
by sharing a core with other minor threads can be avoided. Furthermore, since the
dominant thread account for a significant portion of CPU utilization,
a migration of the dominant thread from a core to another can fluctuate
the utilization of cores, which makes utilization less stable.
In Algorithm~\ref{algo:tlpdvfs}, {\tt part 2} shows the dominant thread scheduling
({\tt DT-aware scheduling}).

%\begin{algorithm}[t]
\begin{algorithm}[t]
\caption{TLP-aware DVFS and scheduling}
\label{algo:tlpdvfs}
\begin{algorithmic}
{\footnotesize
\STATE // initially, $headroom \gets 10$,
\STATE // $inc\_headroom \gets 0$, $dec\_headroom \gets 0$
\STATE
\STATE // part 1: check the usefulness of sample
\IF {$core\_util == 100\%$ for any core}
\STATE set the maximum frequency
\STATE set\_scheduler\_timer(10ms)
\RETURN
\ENDIF
\STATE
\STATE // part 2: DT-aware scheduling
\FOR {each thread}
\STATE {\bf if} $thread\_util >= DOMINANT\_THRESHOLD$ %XXX
\STATE \quad mark as a dominant thread
\ENDFOR
\STATE allocate one core for each dominant threads
\STATE
\STATE // part 3: Adjust headroom 
\STATE $freq\_util \gets \sum_{i=0}^{num\_cores} core\_freq_i * core\_util_i$
\STATE $history$.append($freq\_util$)
\STATE $stdev$ = standard deviation of the last 16 entries of $history$
\STATE {\bf if} $stdev > LARGE\_STDEV$
\STATE \quad {\bf then} $inc\_headroom++$; $dec\_headroom \gets 0$
\STATE {\bf else if} $stdev < SMALL\_STDEV$
\STATE \quad {\bf then} $dec\_headroom++$; $inc\_headroom \gets 0$
%\STATE {\bf else}
%\STATE \quad {\bf then} $dec\_headroom \gets 0$; $inc\_headroom \gets 0$
\STATE
\STATE {\bf if} $inc\_headroom > INC\_HEADROOM\_THRESHOLD$
\STATE \quad {\bf then} $headroom++$
\STATE {\bf if} $dec\_headroom > DEC\_HEADROOM\_THRESHOLD$
\STATE \quad {\bf then} $headroom--$
\STATE
\STATE // part 4: thread packing
\STATE $best\_core \gets 0$, $best\_freq \gets 0$
\STATE $min\_power \gets VERT\_LARGE\_NUMBER$
\FOR {$i \gets 1 ... num\_cores$}
\STATE Divide $thread\_util$s to $i$ cores to as evenly as possible
%\STATE $largest\_util \gets$ largest utilizatoin among cores
\STATE $freq \gets$ the smallest frequency the largest utilization
\STATE \quad\quad\quad\quad will be less than $(100-headroom)$
\STATE $power \gets$ power\_model($freq$, $thread\_util$ distribution)
\IF {$power < min\_power$}
\STATE $best\_core \gets i$, $best\_freq \gets freq$
\STATE $min\_power \gets power$
\ENDIF
\ENDFOR
\STATE
\STATE pack threads to $best\_core$ and set the frequency as $best\_freq$
\STATE set\_scheduler\_timer(100ms)
\RETURN
}
\end{algorithmic}
\end{algorithm}


\section{Dynamic Headroom Adjustment}

\begin{figure}[bt]
\begin{center}
\epsfig{figure=graphs/app_stddev.eps, width=7cm}
\vspace{-0.2in}
\end{center}
\caption{Standard Deviation Difference}
\label{fig:stdv_phase}
\end{figure}

The second technique adjusts the headroom for DVFS. In the original on-demand 
governor as shown in Algorithm~\ref{algo:ondemand}, when the current utilization is higher than
{\tt up-threshold}, the frequency is increased to the max frequency to enhance the user responsiveness.
However, it can unnecessarily increase the frequency, as even from the lowest 200MHz,
the frequency can be increased to the peak frequency.

The second technique attempts to maintain the highest possible utilization by
setting the clock frequency as low as possible. Unlike the on-demand governor 
policy, the clock frequency is increased by steps, even at high utilization.
However, a problem with such
a narrow margin in utilization is that it may degrade performance when the
utilization fluctuate significantly. To mitigate the effect, the scheme adds 
a small percentage of headroom when it sets the clock frequency.

The headroom for clock frequency setting is adjusted dynamically.
The interactive and responsive phases have different requirements for the 
frequency headroom. The interactive phases tend to have stable utilizations regardless
of user inputs, so they do not require a high headroom, allowing to reduce 
frequency at the lowest level. 
However, responsive phases exhibit high fluctuation of utilization,
and thus a large headroom is required not to degrade user responsiveness for a sudden surge
of CPU utilization. Figure~\ref{fig:stdv_phase} presents the standard deviation of
application loads for the two classes of applications. The figure shows that
the interactive applications exhibit much less fluctuation of utilization, and
such stable utilization allows a small headroom for frequency setting.

\begin{comment}
\begin{table}[tb]
\begin{center}
%\begin{footnotesize}
\begin{tabular}{r|r}
\hline \hline
Threshold       & Probability   \\ \hline
80      &       0.003   \\
85      &       0.013   \\
90      &       0.003   \\
95      &       0.009   \\
\hline \hline
\end{tabular}
%\end{footnotesize}
\end{center}
%\vspace*{-0.15in}
\caption{Probability of peak utilization with the interative applications}
\label{tab:prob_peak}
%\vspace*{-0.2in}
\end{table}

To show the utilization stability of interative applications, Table~\ref{tab:prob_peak} presents
the probability of peak utilization for the interative applications.
It checks whether CPU utilization reaches 99-100\%, meaning the current
frequency setting can be too low for the current CPU load. Such a low frequency setting can degrade the application
performance, as the application cannot receive enough CPU resource.
It uses the on-demand governor, and uses four different {\tt up-threshold} settings from the conservative 80\% 
to the most aggresive 95\%. As shown by the table, the utility of interactive applications do not require
the conservative policy by the on-deman governor, as with even with a very high threadhold of 95\%, the probability
\end{comment}

As shown in Algorithm~\ref{algo:tlpdvfs} ({\tt part 2}),
the power manager maintains the past 16 samples of utilization.
If the variation of the utilization samples is higher than a threshold, the headroom
is increased, expecting responsive phases. If the variation stays low, the headroom
is decreased to reduce unnecessary frequency increase.

\section{Thread Packing}

The final technique is to determine the most energy efficient number of active
cores, and pack threads to the determined number of cores. A dominant thread is
scheduled to a separate core, and the other non-dominant threads are packed 
into the rest of cores. 
Finding an optimal number of cores and packing threads on the cores have been
proposed by Cochran et al. \cite{packandcap}  However, our method uses 
a much simpler power model, since a utility-based model is good enough for
mobile applications, unlike compute-intensive multi-threaded applications used
by the prior study.

To determine the optimal number of cores, we use the utility-based power model, as discussed
in Section 2.2. Using the power model, for the current utilization for active threads,
the best number of cores and frequency is selected by searching all possible combinations
of power for different frequency and core counts. In {\tt part 4} of Algorithm~\ref{algo:tlpdvfs},
the power model is used to select the best number of cores and the minimum frequency 
to satisfy the current utilization with the frequency headroom determined by the headroom
adjustment ({\tt part 2}).
The proposed technique can be used regardless of the support for per-core DVFS. 



% 5. Result
\section{Experimental Results}

\section{Methodology}
In this section, we describe our experimental methodology. 
Our target device is a state-of-a-art smart phone which has the Exynos4412 CPU with four cores.
The device runs the Android 4.0.4 operating system which includes Dalvik Virtual Machine (VM). 
All applications run on an instance of their Dalvik VM. 
The Moonsoon power meter measures the power consumption for the whole system that uses a single lithium (Li) battery. 
To evaluate the effectiveness of our power management scheme, we use power, energy, and $ED^2P$ (Energy*Delay*Delay) metrics to assess
the differnt aspects of the proposed schemes.

Although, for the benchmark application shown in Table~\ref{tab:app_summary}, we measure the power consumption, the execution
delay is measured only for the responsive applications. The execution time of an interactive application is determined by
how long a user uses the application, and it is not directly affected by CPU performance. Instead, if the performance
is degraded, the user will perceive the slowdown of the execution, for example slow changes of game scenes etc.
In the paper, to show that the proposed scheme does not incur any perceivable performance degradation, we 
conduct a user study with a 13 user sample for the four benchmark applications with interactive phases.

For the responsive applications, we define a delay as the time period from the initation of a user input to the completion
of the initiated job. As the delay is shortened, the processor can become idle sooner, reducing the CPU energy.
To measure the delay for the web browser and mobile chrome, 
we use BBench which is a web-page rendering benchmark with 11 of the popular sites such as Amazon, BBC, and eBay etc \cite{characterinteractive}. 
The BBench shows page rendering time for each of the pages. 
To measure the delay for the other responsive applications in a repeatable manner, we use the MonkeyRunner tool which 
provides the emulation and automation of diverse user actions such as touch, drag, etc \cite{monkeyrunner}.
The Google Maps for the android application selects search button, types a popular city address, and get directions. 
The search is repeated 5 times with different city addresses. 
The photo editor application selects one of photos, hits the touch effect button, and apply different 12 effects. 
The ezPDF Reader opens different 8 files with the pdf file format.

We evaluate three configurations, including dominant thread scheduling ({\tt DT scheduling}), 
phase-aware headroom adjustment ({\tt phase-aware DVFS}), and thread packing ({\tt packing}). The baseline system uses
the on-demand governor, and all the results are normalized to those with the baseline configuration. 

%we start the real mobile application applied the MonkeyRunner which emulated user diverse action and automated a series of user interaction
%for example, typing a specific city address, touching a button to enter next step, and dragging the touch screen. 
%Third, the application runs by user interaction, step by step, when CPUs become idle state. Finally, 
%all of interactive action is finished and display execution delay of the application.



\subsection{Power}

\begin{figure}[bt]
\begin{center}
\epsfig{figure=graphs/interactive_power_result.eps, width=8cm}
%\vspace{-0.2in}
\end{center}
\caption{Power saving percentage normalized to Ondemand governor}
\label{fig:interactive_power_result}
\end{figure}

\begin{figure}[bt]
\begin{center}
\epsfig{figure=graphs/responsive_power_result.eps, width=8cm}
%\vspace{-0.2in}
\end{center}
\caption{Power saving percentage normalized to Ondemand governor}
\label{fig:responsive_power}
\end{figure}

We first evaluate the power saving with the proposed techniques. Figure~\ref{fig:interactive_power_result} and~\ref{fig:responsive_power} 
present the power saving compared to the baseline on-demand governor, for the interactive, and responsive applications respectively. 
For the interactive applications, the proposed techniques effectively reduce power up-to 20\% for {\tt angry bird}. 
The moive player has little improvement, since the on-demand governor can set the frequency effectively for the application with
a low and stable utilization. Among the three techniques, the interactive applications benefit most from the phase-aware headroom
adjustment. Since the utilization is relatively stable, the tight frequency setting can reduce the overall power, without
hurting user interactivity, as will be discussed with a user study in Section 5.6. 

For the responsive applications, three techniques have a relatively small power saving, or increase power for the two applications.
The headroom adjustment cannot reduce the headroom for the responsive applications effecitvely, since they have a high fluctuation of 
utlization, and thus require a large headroom not to degrade the performance and system responsiveness. 

\subsection{Delay, Energy, and ED\textsuperscript{2}P for the responsive applications}

\begin{figure}[bt]
\begin{center}
\epsfig{figure=graphs/responsive_latency_result.eps, width=8cm}
%\vspace{-0.2in}
\end{center}
\caption{Latency Improvement percentage normalized to Ondemand governor}
\label{fig:responsive_latency_result}
\end{figure}

\begin{figure}[bt]
\begin{center}
\epsfig{figure=graphs/responsive_energy_result.eps, width=8cm}
%\vspace{-0.2in}
\end{center}
\caption{Energy saving percentage normalized to Ondemand governor}
\label{fig:responsive_energy_result}
\end{figure}

Although the proposed scheme does not provide large power saving for the responsive applications, DT-ware scheduling and
thread packing is quite effective in reducing the delay. Figure~\ref{fig:responsive_latency_result} shows 
the normalized delay for the responsive applications. Among the three techniques, fixing a dominant thread to
a core reduces the delay significantly for the responsive applications. A dominant thread can use
caches without any pollution from other threads, and do not need to be migrated between cores.
Figure~\ref{fig:responsive_energy_result} shows the energy saving for the responsive applications. 
Since the delay is significantly reduced by the proposed scheduing scheme, the actual energy consumption
is reduced, even if the power may increase slightly as shown in Figure~\ref{fig:responsive_power}.
The overall energy saving can be from 5\% to 10\% across the 5 responsive applications.

\begin{figure}[bt]
\begin{center}
\epsfig{figure=graphs/responsive_EDP_result.eps, width=8cm}
%\vspace{-0.2in}
\end{center}
\caption{ED\textsuperscript{2}P Improvement percentage normalized to Ondemand governor}
\label{fig:responsive_EDP_result}
\end{figure}

Figure~\ref{fig:responsive_EDP_result} shows the ED\textsuperscript{2}P improvements by the proposed technique.
For the responsive applications, the fast system response is critical for user experience in mobile platforms,
as the increasing computational power with multi-cores is supposed to improve the system responsiveness.
In the figure, ED\textsuperscript{2}P improvements are signifcant from 10-33\% for the four applications,
except for the pdf reader.

\subsection{System Reponsiveness}

\begin{figure}[bt]
\begin{center}
\epsfig{figure=graphs/interactive_UX_result.eps, width=8cm}
%\vspace{-0.2in}
\end{center}
\caption{User satisfaction test normalized to Ondemand governor}
\label{fig:interactive_UX_result}
\end{figure}

To assess the system responsiveness for the four interactive applications, we conduct a user study.
We asked 13 users to evaluate the effect of our techniques on the user satisfaction. 
A volunteer uses applications with ondemand governor, assuming the governor satisfy all users, and experience the application execution first 
with the baseline governor. 
After that, the user tries the same application with the proposed governor without any notification 
to identify whether the volunteers detect potential poor interactivity.  
Figure~\ref{fig:interactive_UX_result} shows the results of our user study for four applications. 
Only one user reports a minor slowdown for a game, among 13 users for four applications.
The results show many users do not sense the performance loss for the majority of applications.

\begin{comment}
\begin{figure}[bt]
\begin{center}
\epsfig{figure=graphs/latency_different_headroom_size.eps, width=7cm}
\vspace{-0.2in}
\end{center}
\caption{Latency between user input and frequency raising}
\label{fig:latency_different_headroom_size}
\end{figure}

The second method for system responsiveness is to check whether CPU utilization reaches 99-100\%, meaning the current
frequency setting can be too low for the current CPU load. Such a low frequency setting can degrade the application 
performance, as the application cannot receive enough CPU resource.

\end{comment}


\section{Related Work}

There have been several recent studies for analyzing smartphone applications~\cite{anatomizing, WhySlowBrowser, tmapp, characterinteractive}.
Some work focus on web-browsing applications, with automation of user inputs for deterministic analysis~\cite{anatomizing, WhySlowBrowser}.
Issa et al. and Gutierrez et al. propose a benchmark suite for smartphones,
and characterize their system and architectural behaviors~\cite{tmapp, characterinteractive}. 
To find representative applications, the smartphone usage pattern has been studied~\cite{diversity,intothewild,diversebehavior}.
The prior work log system activities for smartphones, to investigate the real world usage patterns.
The analysis of TLP for interactive applications have been studied by Blake et al~\cite{desktopTLP}. They
analyzed the available TLP for common desktop applications. We share a similar method to
investigate the TLP of smartphone applications. However, unlike the desktop analysis,
we emphasize the importance of interactive and responsive phases in mobile applications,
and show the significant skewed resource usage among threads.

Since the battery-life is a crucial problem of smartphones,
the power consumption pattern of smartphone has been widely studied.
Carroll and Gernot analyze a breakdown of power distribution of each component in smartphone~\cite{smartphonepower}.
They show that the communication module and display consume a significant portion of power.
There have been several studies to model the power consumption of mobile platforms.
Shye et al.  propose per-component power estimation methods, using the linear regression to build a model~\cite{intothewild}.
Yoon et al. proposes an energy metering framework for mobile platforms by monitoring kernel activities~\cite{AppScope}.
Pathak et al. use system call tracing~\cite{powermodelsyscall} to model the power consumption.
and some studies generate power model online using battery voltage sensors~\cite{selfconstructive, accurateonline}. 
In our work, we use a simple CPU power model similar to the ones used by Ma et al. and Kansal etal~\cite{isca11DVFS, VMpower}.

There have been a significant body of work for energy saving with the DVFS mechanism~\cite{micro06DVFS, decomposition, Koala, isca11DVFS, packandcap}. 
Many studies target non-interactive server and desktop platforms with application continuously using CPUs with 100\% utilization.
The studies maximizes the energy efficiency or the throughput under a power constraint.
There are also DVFS mechanisms for interactive applications.
\cite{transactionbased} distinguishes the interactive task from the background tasks,
and keep user-perceived latency less than human-perceptual threshold.
\cite{IADVS} aims at similar timing constraint, 
while it is based on the fine-grained interaction history.
\cite{intothewild} directly target the smartphone applications.
They exploit the human perception study 
which indicates that human hardly detect large changes in their surrounding environment.
They down CPU frequency as slowly as human cannot detect, so that the energy is reduced while user rarely detect.
Compared to them, our work exploits uneven thread level parallelism of smartphone applications.
In addition, it keeps the CPU utilization under 100\%.
Thus, the performance rarely degrades and so does the user responsiveness.


%\vspace{-0.2in}
\section{Conclusions}
%\vspace{-0.2in}

This paper characterizes the power behaviors of smartphone multi-cores and interactive applications. 
Although the number of cores are increasing, the available TLP is still limited in the interactive
applications. Furthermore, the available TLP is highly skewed, with one or two threads using
a significant portion of CPU utilization. We classified main applications phases into
interactive and responsive phases, and showed that the two phases have different usage patterns
and responsive requirements.
Based on the analysis, we proposed a power management technique with three components,
dominant thread scheduling, phase-aware headroom adjustment, and thread packing. The interactive
applications benefit most from the phase-aware headroom adjustment due to their stability
of CPU usages. Although the proposed technique does not save power significantly for
the responsive applications, the scheduling scheme reduces the execution delay significantly,
and thus improves energy and $ED^2$ for the responsive applications. 





\begin{comment}
%%
%% 표 삽입 예시
%% Example. how to insert table
%%
\begin{table}[t]
\caption{Energy stability $E$ (eV) per molecule of all meta-stable
isomer states of C$_{60}$ opening process for forming the (5,5) cap.
In the SW-I and SW-II, both ferromagnetic (Ferro) and paramagnetic (Para)
spin configurations are obtained, whereas only non-magnetic configuration
is obtained in the BF, SW-III, and CAP(5,5).
$M$ is total magnetization $n_{\rm up}$-$n_{\rm down}$ in unit of $\mu_B$, where
$n_{\rm up(down)}$ is the number of up (down) spins.
}
\label{mag-tab1}
\begin{center}
\begin{tabular} {ccccccccccc}
\hline\hline
& & BF &\multicolumn{2}{c}{SW-I}&&\multicolumn{2}{c}{SW-II}&SW-III&CAP&\\
\cline{4-5} \cline{7-8}
&               &   &  Para & Ferro &&   Para &  Ferro &      &      &\\
\hline
& $E$ (eV)      & 0 & 7.796 & 7.832 && 10.418 & 10.408 & 11.5 & 13.2 &\\
& $M$ ($\mu_B$) & 0 &     0 &  1.94 &&      0 &   2.06 &    0 &    0 &\\
\hline\hline
\end{tabular}
\end{center}
\end{table}

Energy stabilities are summarized at Table \ref{mag-tab1}.
Except SW-I and SW-II where zigzag-type atoms appear, non-magnetic configuration
is favorable even in the presence of undercoordinated atoms.
The dangling bond in armchair-type atoms is passivated by forming a double
bond between them, of which length is about 1.239 {\AA}.
Due to the slight difference of arrangement of zigzag-type atoms, SW-I
favors a paramagnetic (Para) configuration in amount of 36 meV,
while SW-II does a ferromagnetic (Ferro) configuration in gain of 10 meV.
The net magnetic moment corresponds to two unpaired electrons.

%%
%% 그림 삽입 예시
%% Example. how to insert graph
%%
%% Note. 가급적 \includegraphics 명령을 사용하십시오.
%% Recommen : Use \includegraphics to insert graph.
%%
\begin{figure}[t]
    \centerline{\includegraphics[width=12.5cm]{sample-fig1}}
    \caption{ Ball-and-stick models of meta-stable isomers in
        cage opening process from a C$_{60}$ buckminsterfullerene
        to a (5,5) capsule. We name them BF and CAP(5,5).
        Depending on the number of the Stone-Wales (SW) transformation,
        we call the intermediate isomers with SW-I, SW-II, and SW-III.
        Highlighted atoms are undercoordinated except BF.
    } \label{mag-fig1}
\end{figure}

\end{comment}
%%
%% 참고문헌 시작
%% Refences
%%
%\input{reference}

\bibliographystyle{acm}
\bibliography{references}

\begin{comment}
\begin{thebibliography}{00}

\bibitem{Iijima91} S. Iijima,
         Nature (London) {\bf 354}, 56 (1991).
         %Helical microtubules of graphitic carbon.

\bibitem{Dresselhaus96} M. S. Dresselhaus, G. Dresselhaus, and P. C. Eklund.
         {\em Science of Fullerenes and Carbon Nanotubes} (Academic, San Diego, 1996).

\bibitem{Saito98} R. Saito, G. Dresselhaus, and M. S. Dresselhaus,
         {\em Physical Properties of Carbon Nanotubes}
         (Imperial College Press, London, 1998).

\bibitem{Makarova01} T.L. Makarova, B. Sundqvist,
         R. H\"ohne, P. Esquinazi, Y. Kopelevich, P. Scharff,
         V.A. Davydov, L.S. Kashevarova, and A.V. Rakhmanina,
         Nature (London) {\bf 413}, 716 (2001).

\bibitem{Palacio01} F. Palacio, Nature (London) {\bf 413}, 690 (2001),
         and references therein.

\bibitem{SW} A.J. Stone and D.J. Wales,
         Chem. Phys. Lett {\bf 128}, 501 (1986).

\end{thebibliography}
\end{comment}

%%
%% 한글요약문 시작 (Korean summary)
%%
%% Note. 영문논문일 경우에만 필요하니 한글논문의 경우에는 작성하지 마십시오.
%%
\begin{summary}

스마트폰이 확산됨에 따라 스마트폰과 그 애플리케이션의 전력 소모 방식에 대한 이해는 
배터리 용량의 한계를 극복하는데 필수요소가 되었다. 
하지만 스마트폰에서 동작하는 애플리케이션은 일반적으로 사용자의 상호 동작의 단계에 따라 다르며, 
이것은 기존의 전통적인 CPU 밴치마크 애플리케이션과는 다른 특성을 보인다. 
본 연구에서는 인터랙티브 애플리케이션의 수행 단계를 구분하고, CPU 사용 패턴과 쓰레드 수준의 동시성의 특징들을 분석하고, 이용률 기반의 전력 모델을 구성하였다.

분석된 내용을 기반으로, 본 연구에서는 동적 전압 및 주파수 스케일링을 이용한 개선된 전력 관리 기법을 제안하였다. 
제안된 기법에서는 모바일 애플리케이션의 사용자 체감 성능에 절대적인 영향을 끼치는 '성능 지배적 쓰레드'를 찾아 별도로 관리되도록 하였다. 
또한 이용률 기반의 전력 모델을 이용하여, 멀티 코어 환경에서 최적으로 에너지 효율적인 실행 코어의 개수를 찾아내고, 
'성능 지배적 쓰레드'를 제외한 다른 쓰레드를 동적으로 통합하고 분리하는 방식으로 스케줄링 되도록 하였다. 
마지막으로 체감 성능의 저하 방지를 위하여 주파수 조절 시 두었던 여분의 공간을 사용자의 입력 단계에 맞추어 동적으로 조절하도록 하였다. 
제안된 기법을 안드로이드 스마트폰에 구현하고, 인터랙티브 애플리케이션에 대해서 실험한 결과, 
대부분의 경우 사용자의 반응성을 해치지 않고 Ondemand 정책 대비 평균 에너지는 7\%, ED\textsuperscript{2}P는 17\% 절감할 수 있음을 보였다.
\end{summary}


%%
%% 감사의 글 시작
%% Acknowledgement
%%
% @command acknowledgement 감사의글
% @options [default: 클래스 옵션 korean|english ]
% - korean : 한글타이틀 | english : 영문타이틀

\acknowledgement[korean]

    이 논문을 완성하기까지 주위의 모든 분들로부터 수많은 도움을 받았습니다.

    끝으로 오늘의 제가 있을 수 있도록 사랑으로 키워 주신
    어머니와 또한 가족들에게 감사드립니다.
    저의 이 작은 결실이 그분들께 조금이나마 보답이 되기를 바랍니다.

%%
%% 이력서 시작
%% Curriculum Vitae
%%
% @command curriculumvitae 이력서
% @options [default: 클래스 옵션 korean|english ]
% - korean : 한글이력서 | english : 영문이력서
\curriculumvitae[korean]

    % @environment personaldata 개인정보
    % @command     name         이름
    %              dateofbirth  생년월일
    %              birthplace   출생지
    %              domicile     본적지
    %              address      주소지
    %              email        E-mail 주소
    % - 위 6개의 기본 필드 중에 이력서에 적고 싶은 정보를 입력
    % input data only you want
    \begin{personaldata}
        \name       {우 상 정}
        \dateofbirth{1980}{6}{2}
        \address    {경기도 수원시 영통구 영통동 신나무실 신안아파트 533동 507호}
        \email      {sjwoo@calab.kaist.ac.kr}
    \end{personaldata}

    % @environment education 학력
    % @options [default: (none)] - 수학기간을 입력
    \begin{education}
        \item[1996. 3.\ --\ 1999. 2.] 홍익대학교 사범대학 부속고등학교 
        \item[1999. 3.\ --\ 2007. 2.] 인하대학교 컴퓨터공학과 (B.S.)
        \item[2011. 2.\ --\ 2013. 1.] 한국과학기술원 전산학과 (M.S.)
    \end{education}

    % @environment career 경력
    % @options [default: (none)] - 해당기간을 입력
    \begin{career}
	\item[2005. 7.\ --\ 2006.12] (주)삼성전자 소프트웨어멤버십
	\item[2007. 1.\ --\ 현재] (주)삼성전자 DMC 연구소 연구원
    \end{career}

    % @environment activity 학회활동
    % @options [default: (none)] - 활동내용을 입력
    \begin{activity}
	\item \textbf{Sangjung Woo}, Wonik Seo, Changdae Kim, and Jaehyuk Huh,
	     \textit{User Input based Power Reduction Technique for Smartphones}, 
	     '11 Proceedings of the Korea Computer Congress (Best paper award)
    \end{activity}
    
    % @environment publication 연구업적
    % @options [default: (none)] - 출판내용을 입력
    \begin{publication}
	\item \textbf{Sangjung Woo}, Wonik Seo, Changdae Kim, and Jaehyuk Huh,
	      \textit{User Input based Power Reduction Technique for Smartphones},
	      Journal of Computing Science and Engineering (JCSE)
    \end{publication}

%% 본문 끝
\end{document}
