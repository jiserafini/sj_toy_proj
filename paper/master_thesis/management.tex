\section{Current Power Management Schemes}

In this section, we analyze the behaviors of a state-of-art power management module using DVFS.
The target smartphone platform uses the {\tt interactive} governor which enhances the {\tt ondemand} governor
available in plain Linux platforms. For a single core, the interactive governor reduces the clock
frequency to reduce under-utilization of cores. However, when a user touch occurs, the clock frequency
is maximized not to delay the process of a user interaction. The clock frequency is slowly decreased once
the process of user interaction is completed and CPU utilization becomes low at the maximum clock frequency.

The CPU hot plug-in mechanism is also combined with the DVFS power management scheme. The second core is turned on,
if the frequency of the first core, which is adjusted by the aforementioned governor, is higher 
than a threshold more than N times. N is often set to five. The second core is turned off, if the second core 
frequency is lower than a threshold more than N times. In the current mechanism, the decision is made based
on indirect information of TLP, set by the power governor. It may not fully consider the available
TLP and the balance of such TLP.

\begin{figure}[bt]
\begin{center}
\epsfig{figure=graphs/wasted_cycles.eps, width=7cm}
%\vspace{-0.2in}
\end{center}
\caption{Wasted cycles: fixed 1.2GHz, ondemand, and interactive governor with two cores}
%\vspace{-0.2in}
\label{fig:wasted_cycles}
\end{figure}

To evaluate the potential improvement for better power management, we analyze the wasted CPU cycles with
the current power governors and hot plug-in schemes. Figure~\ref{fig:wasted_cycles} presents the
percentage of wasted CPU cycles for two cores for three configurations: fixed 1.2GHz, ondemand,
and interactive governors. The wasted cycles are defined as the sum of unused CPU cycles due to
less than 100\% utilization. To optimize the energy efficiency, the wasted cycles must be
eliminated to provide the exact amount of computational power to interactive applications by
DVFS.  

The first bar shows the wasted CPU cycles when the clock frequency of both cores is fixed to
1.2GHz. For six applications, more than 70-95\% of clock cycles are wasted, since the actual
CPU utilization is quite low, as shown in Table~\ref{tab:summary}. 
The wasted cycles can be eliminated if an ideal governor can change the frequency arbitrarily. 
However, there are three potential problems with having an ideal governor. Firstly,
the governor cannot predict the future utilization accurately without delaying applications
by aggressive frequency reduction. Secondly, per-core DVFS is not supported in the current
mobile multi-cores, so that clock frequencies
of multiple cores must be set to the same frequency. Finally, the clock frequency can be changed
only by discrete steps.

The second and third bars show the wasted cycles with the current two governors. 
Compared to the fixed 1.2GHz configuration, both governors have significant amounts of
wasted cycles. Compared to an ideal system, the current governors have a significant room for further 
improvement for better DVFS-based power management. However, to adjust frequencies more aggressively
without affecting users' perception on the system performance, it will be 
critical to accurately model the utilization and TLP for interactive applications. 