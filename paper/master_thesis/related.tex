\chapter{Related Work}

There have been several recent studies for analyzing smartphone applications~\cite{characterinteractive,anatomizing, tmapp, WhySlowBrowser}.
Some work focus on web-browsing applications, with automation of user inputs for deterministic analysis~\cite{anatomizing, WhySlowBrowser}.
Issa et al. and Gutierrez et al. propose a benchmark suite for smartphones,
and characterize their system and architectural behaviors~\cite{characterinteractive, tmapp}. 
To find representative applications, the smartphone usage pattern has been studied~\cite{diversity,intothewild,diversebehavior}.
The prior work log system activities for smartphones, to investigate the real world usage patterns.
The analysis of TLP for interactive applications have been studied by Blake et al~\cite{desktopTLP}. They
analyzed the available TLP for common desktop applications. We share a similar method to
investigate the TLP of smartphone applications. However, unlike the desktop analysis,
we emphasize the importance of interactive and responsive phases in mobile applications,
and show the significant skewed resource usage among threads.

Since the battery-life is a crucial problem of smartphones,
the power consumption pattern of smartphone has been widely studied.
Carroll and Gernot analyze a breakdown of power distribution of each component in smartphone~\cite{smartphonepower}.
They show that the communication module and display consume a significant portion of power consumption.
There have been several studies to model the power consumption of mobile platforms.
Shye et al.  propose per-component power estimation methods, using the linear regression to build a model~\cite{intothewild}.
Yoon et al. proposes an energy metering framework for mobile platforms by monitoring kernel activities~\cite{AppScope}.
Pathak et al. use system call tracing~\cite{powermodelsyscall} to model the power consumption.
and some studies generate power model online using battery voltage sensors~\cite{selfconstructive, accurateonline}. 
In our work, we use a simple CPU power model similar to the ones used by Ma et al. and Kansal etal~\cite{VMpower, isca11DVFS}.

There have been a significant body of work for energy saving with the DVFS mechanism~\cite{decomposition, packandcap, micro06DVFS, isca11DVFS, Koala}. 
Many studies target non-interactive server and desktop platforms with application continuously using CPUs with 100\% utilization.
The studies maximizes the energy efficiency or the throughput under a power constraint.
There are also DVFS mechanisms for interactive applications.
\cite{transactionbased} distinguishes the interactive task from the background tasks,
and keep user-perceived latency less than human-perceptual threshold.
\cite{IADVS} aims at similar timing constraint, 
while it is based on the fine-grained interaction history.
\cite{intothewild} directly target the smartphone applications.
They exploit the human perception study 
which indicates that human hardly detect large changes in their surrounding environment.
They down CPU frequency as slowly as human cannot detect, so that the energy is reduced while user rarely detect.
Compared to them, our work exploits uneven thread level parallelism of smartphone applications.
In addition, it keeps the CPU utilization under 100\%.
Thus, the performance rarely degrades and so does the user responsiveness.
