\begin{table}[tb]
\begin{center}
\begin{footnotesize}
\begin{tabular}{l|l|r}
\hline \hline
\multicolumn{2}{c|}{Mode}	& 	latency(msec)	\\
\hline
\multirow{2}{*}{Hotplugging}	&	On	&	17.251 \\\cline{2-3}
				&	Off	& 4.346 \\
\hline
\multicolumn{2}{c|}{DVFS}	&	1.045	\\
\hline \hline
\end{tabular}
\end{footnotesize}
\end{center}
%\vspace*{-0.15in}
\caption{Latency of Exynos 4412 AP}
\label{tab:exynos_latency}
% %\vspace*{-0.2in}
\end{table}

\begin{comment}
\multirow{2}{*}{Idle}		&	Clock gating	&	0.001 \\ \cline{2-3}
				&	HW power down	& 	0.300 \\

[Clock gating(WFI(Wait For Interrupt))]

WFI and WFE Standby modes disable most of the clocks of a processor, while keeping
its logic powered up. This reduces the power drawn to the static leakage current, leaving
a tiny clock power overhead requirement to enable the device to wake up.
Entry into WFI Standby mode is performed by executing the WFI instruction.


[ARM Power down mode]

Shutdown mode powers down the entire device, and all state, including cache, must be saved
externally by software. This state saving is performed with interrupts disabled, and finishes with
a Data Synchronization Barrier operation. The Cortex-A9 processor then communicates with a
power controller that the device is ready to be powered down in the same manner as when
entering Dormant Mode. The processor is returned to the run state by asserting reset.
\end{comment}
