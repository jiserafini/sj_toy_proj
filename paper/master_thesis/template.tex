\documentclass[pageno]{jpaper}

%replace XXX with the submission number you are given from the ISCA submission site.
\newcommand{\iscasubmissionnumber}{XXX}

\usepackage[normalem]{ulem}

\begin{document}

\title{
Guidelines for Submissions to ISCA 2013}

\date{}
\maketitle

\thispagestyle{empty}

\begin{abstract}
This document is intended to serve as a sample for submissions to ISCA 2013.
We provide some guidelines that authors should follow when submitting papers to
the conference.
\end{abstract}

\section{Introduction}

This document provides the formatting instructions for submissions to the 40th
Annual International Symposium on Computer Architecture,
2013. In an effort to respect the efforts of reviewers and in
the interest of fairness to all prospective authors, we request that all
submissions to ISCA-40 follow the formatting and submission rules detailed
below.  Submissions that violate these instructions may not be
reviewed, at the discretion of the program chair, in order to maintain a review
process that is fair to all potential authors.  This is a generous format,
with plenty of space -- there should be no need to tweak it in any
significant way.

An example submission (formatted using the ISCA-40 submission format) that
contains the submission and formatting guidelines can be downloaded from
the conference website at {\em http://isca2013.eew.technion.ac.il}.

All questions regarding paper formatting and submission should be directed to
the program chair.

\section{Preparation Instructions}

\subsection{Paper Formatting}

All submissions should contain a maximum of 12 pages of single-spaced
two-column text. If you are using \LaTeX~\cite{lamport94} 
to typeset your paper, then we 
strongly suggest
that you use the template available at
http://isca2013.eew.technion.ac.il -- this
document was prepared with that template.  
If you are using a different
software package to typeset your paper, then please adhere to the guidelines
given in Table~\ref{table:formatting}.

\begin{table}[h!]
  \centering
  \begin{tabular}{|l|l|}
    \hline
    \textbf{Field} & \textbf{Value}\\
    \hline
    \hline
    Page limit & 12 pages\\
    \hline
    Paper size & US Letter 8.5in $\times$ 11in\\
    \hline
    Top margin & 1in\\
    \hline
    Bottom margin & 1in\\
    \hline
    Left margin & 0.75in\\
    \hline
    Right margin & 0.75in\\
    \hline
    Separation between columns & 0.25in\\
    \hline
    Body font & 10pt\\
    \hline
    Abstract font & 10pt, italicized\\
    \hline
    Section heading font & 12pt, bold\\
    \hline
    Subsection heading font & 10pt, bold\\
    \hline
    Caption font & 9pt, bold\\
    \hline
    References & 8pt\\
    \hline
  \end{tabular}
  \caption{Formatting guidelines for submission. }
  \label{table:formatting}
\end{table}

\textbf{Please ensure that you include page numbers with your
submission}. This makes it easier for the reviewers to refer to
different parts of your paper when they provide comments.

Also, please ensure that your submission has a banner at the top of 
the title page, similar to this one, which contains the submission
number and the notice of confidentiality.  If using the template,
just replace XXX in the template with the submission number
you receive from the submission website.

If you use bibtex, please note that the references.bib file provided in the template example includes some format-specific incantations at the top of the file.  If you substitute your own bib file, you will probably want to include these 
incantations at the top of it.

\subsection{Content}

\noindent\textbf{\sout{Author List.}} All submissions are double
blind. Therefore, please do not include any author names in the
submission. You must also ensure that the metadata included in the
PDF does not give away the authors. If you are improving upon your
prior work, refer to your prior work in the third person and include
a full citation for the work in the bibliography. For example, if
you are building on {\em your own} prior work in the papers \cite{nicepaper,nicepaper2}, 
you would say something like "While the authors of \cite{nicepaper,nicepaper2} did X and Y,
this paper does X, Y and Z, and is therefore much better."  Do NOT omit or
anonymize references for blind review.

\noindent\textbf{Figures and Tables.} Ensure that the figures and
tables are legible.  Please also ensure that you refer to your
figures in the main text. The proceedings will be printed in
gray-scale, and many reviewers print the papers in
gray-scale. Therefore, if you must use colors for your figures, ensure
that the different colors are highly distinguishable in gray-scale.
If a figure is not easily understandable in gray-scale, then assume
it will not be understood by the reviewers.  In many cases, it
is better to just prepare your documents without color.

\noindent\textbf{Main Body.} Avoid bad page or column breaks in
your main text, i.e., last line of a paragraph at the top of a
column or first line of a paragraph at the end of a column. If you
begin a new section or sub-section near the end of a column,
ensure that you have at least 2 lines of body text on the same
column. 

\section{Submission Instructions}

\subsection{Paper Authors}

Declare all the authors of the paper upfront. Addition/removal of authors once
the paper is accepted will have to be approved by the program chair.  The
paper selection process is carefully run in a way that maximizes fairness
by seeking to eliminate all conflicts of interest.  Late changes to author 
lists can invalidate that process.

\subsection{Declaring Conflicts of Interest}

The authors must register all their conflicts into the paper submission site.
Conflicts are needed to resolve assignment of reviewers. 
Please get the conflicts
right.  You have a week between the registration of the paper and final
submission -- there is no need to do the conflicts in a rush at the last
second. If a paper is found to
have an undeclared conflict that causes a problem OR if a paper is found to declare false conflicts in order to abuse or ``game'' the review system, the paper may be rejected.

Please declare a conflict of interest (COI) if any of the following exist for any author of a paper:

\begin{enumerate}
\item Your Ph.D. advisor and Ph.D. students forever.
\item Family relations by blood or marriage forever (if they might be potential reviewers).
\item Other past or current advisors
\item People with whom you collaborated in the last five years. Collaborators include:
\begin{itemize}
\item Co-authors on an accepted/rejected/pending research paper, 
\item Co-PIs on an accepted/rejected/pending grant, 
\item Those who are funders (decision-makers) regarding your research grants, and researchers whom you fund. 
\end{itemize}
You many exclude ``service'' collaborations like writing a CSTB report or serving on a program committee together.
\item People who shared your primary institution in the last five years. Note that if either you or they have moved, there could be several institutions to consider.
\end{enumerate}


There may be others with whom you know a conflict of interest exists.  However,
you will need to justify this conflict of interest.  Please be reasonable.  For
example, just because a reviewer works on similar topics as the paper you are
submitting is on, you cannot declare a conflict of interest with them. The PC Chair may contact co-authors to explain COIs whose origin is not clear.

You will have to declare all conflicts with PC members as well as non-PC members 
with whom you
have a conflict of interest.  When in doubt, contact the program chair.

\subsection{Concurrent Submissions and Resubmissions of Already Published Papers}

By submitting a manuscript to ISCA-40, the authors guarantee that the
manuscript has not been previously published or accepted for publication in a
substantially similar form in any conference or journal. The authors also
guarantee that no paper which contains significant overlap with the
contributions of the submitted paper is under review to any other conference or
journal or workshop, or will be submitted to one of them during the ISCA-40
review period. Violation of any of these conditions will lead to rejection.

Extended versions of papers accepted to IEEE Computer Architecture Letters can
be submitted to ISCA-19.  If you are in doubt, contact the program chair.

Finally, we also note that the ACM Plagiarism Policy ({\em http://www.acm.org/publications/policies/plagiarism\_policy}) covers a range of ethical issues concerning the misrepresentation of other works or one's own work.

\bstctlcite{bstctl:etal, bstctl:nodash, bstctl:simpurl}
\bibliographystyle{IEEEtranS}
\bibliography{references}

\end{document}

