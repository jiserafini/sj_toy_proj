\section{Interactive Application Behaviors}

In this section, we characterize the behaviors of commercial smartphone applications. 
We show how CPU utilization and TLP differ by applications and interactions.
For interactive applications, depending on phases, the behaviors of an application vary 
widely. Therefore, we decompose application phases into three types of phases. Firstly, a {\tt setup}
phase initializes an application, or loads data for a new round of game. Secondly,
a {\tt responsive} phase occurs when cores become significantly active only after a user interaction.
In such a responsive phase, CPUs become idle or show low utilization sometime after the interaction, after
the computation is completed to process the user interaction. A common example of
the responsive phase is the interaction with a web browser application. Once
a hyperlink is touched, the CPUs become active to render a new page. However,
once the process is completed, the CPUs become idle again.

Finally, in an {\tt interactive} phase, the application continuously conducts some work
while users may interact with the application. For some games, such as a car racing game,
CPUs process the game engine continuously regardless of user interactions. At the same time,
user interactions may change the behaviors of such computational work.
An application may exhibit only a subset of the three phases. 

To show the detailed behaviors of applications, Figure~\ref{fig:interact_util_1cpu} presents
the utilization of a single 1GHz core for a game application (Angry-bird) and a web browser.
The dots represent user touch inputs on the touch screen.
For the game application, during
the change of a round, a setup phase initializes the new round. In an interactive phase,
CPU utilization is maintained around 40\%, regardless of user interactions. 
However, in the browser application, only {\tt setup} and {\tt response} phases are mostly
shown. CPU utilization increases only when a user touch occurs, and the utilization
drop soon after the interaction is processed.

\begin{figure}[bt]
\begin{center}
\epsfig{figure=graphs/angry_1cpu.eps, width=8cm}
%\vspace{-0.45in}
\epsfig{figure=graphs/browser_1cpu.eps, width=8cm}
%\vspace{0.01in}
\end{center}
\caption{CPU util. and touch events with 1 core}
%\vspace{-0.2in}
\label{fig:interact_util_1cpu} 
\end{figure}

\begin{figure}[bt]
\begin{center}
\epsfig{figure=graphs/angry_2cpu.eps, width=8cm}
%\vspace{-0.45in}
\epsfig{figure=graphs/browser_2cpu.eps, width=8cm}
%\vspace{0.01in}
\end{center}
\caption{CPU util. and touch events with 2 cores}
%\vspace{-0.2in}
\label{fig:interact_util_2cpu}
\end{figure}

Figure~\ref{fig:interact_util_2cpu} presents the utilization changes with two cores.
For both applications, there are some amount of TLP, although their behaviors are 
quite different. For the game application, the parallelism in two cores differ significantly.
CPU1 exhibits much a higher utilization than CPU0 in interactive phases, showing unbalanced
loads among threads. In the browser application, both cores have similar loads.


\begin{figure}[bt]
\begin{center}
\epsfig{figure=graphs/angry_thread.eps, width=8cm}
%\vspace{-0.45in}
\epsfig{figure=graphs/browser_thread.eps, width=8cm}
%\vspace{0.01in}
\end{center}
\caption{Application threads with 2 cores}
\label{fig:threads}
\end{figure}

Figure~\ref{fig:threads} presents the CPU utilization of active threads with 2 core runs.
For both applications, the top thread dominates core utilization. For the game application,
there is only a small amount of utilization by the rest of threads, causing unbalanced loads
in two cores. For the browser application, there are several other active threads, which
can share the second core.

We apply such phase-based analysis on 10 smartphone applications. For each type of phases,
we show how much computation capability (CPU utilization) the application requires, and how 
much TLP exists.

\begin{table*}[tb]
\begin{center}
\begin{footnotesize}
\begin{tabular}{l|l|l|r|r|r|r|r} 
\hline \hline
\multirow{2}*{App}  &\multirow{2}*{Description}  &\multirow{2}*{Metric}  & \multicolumn{1}{|c}{Setup} & \multicolumn{2}{|c}{Responsive} & \multicolumn{2}{|c}{Interactive} \\
\cline{5-8}
&&  & & \multicolumn{1}{c}{w/o touch} & \multicolumn{1}{|c}{w/ touch} & \multicolumn{1}{|c}{w/o touch} & \multicolumn{1}{|c}{w/ touch}  \\
\hline 
\multirow{3}*{Angry-bird} 	&	\multirow{3}{100pt}{Shooting game with physics engine}
&	CPU util. (\%)		&	73.7	&	29.6	&	47.2	&	39.6	&	42.6	\\\cline{3-8}
&&	length (\%)			&	14.3	&	30.0	&	1.6	&	44.2	&	9.9	\\\cline{3-8}
&&	\# active threads	&	1.6	&	1.2	&	1.4	&	1.3	&	1.5	\\ 
\hline 
\multirow{3}*{Golf}		 	&	\multirow{3}{100pt}{Golf game with 3D graphics}
&	CPU util. (\%)		&	75.9	&	41.3	&	51.3	&	48.5	&	62.4	\\\cline{3-8}
&&	length (\%)			&	10.3	&	18.0	&	2.2	&	67.0	&	2.5	\\\cline{3-8}
&&	\# active threads	&	2.1	&	2.0	&	2.0	&	1.9	&	2.1	\\
\hline 
\multirow{3}*{Trial}	 	&	\multirow{3}{100pt}{Motorcycle racing game with 3D graphics}
&	CPU util. (\%)		&	89.0	&	59.3	&	61.5	&	73.0	&	77.0	\\\cline{3-8}
&&	length (\%)			&	17.6	&	16.4	&	0.8	&	33.3	&	31.9	\\\cline{3-8}
&&	\# active threads	&	2.4	&	2.1	&	1.8	&	2.7	&	2.6	\\
\hline 
\multirow{3}*{Media}		&	\multirow{3}{100pt}{Play a video file}
&	CPU util. (\%)		&	39.0	&	4.6	&	21.3	&	34.8	&	\multicolumn{1}{|c}{-} 	\\\cline{3-8}
&&	length (\%)			&	4.8	&	8.5	&	3.5	&	83.2	&	\multicolumn{1}{|c}{-} 	\\\cline{3-8}
&&	\# active threads	&	1.3	&	0.0	&	0.9	&	1.6	&	\multicolumn{1}{|c}{-} 	\\
\hline 
\multirow{3}*{Browser}	&	\multirow{3}{100pt}{Web-browsing to portal news pages}
&	CPU util. (\%)		&	74.5	&	10.3	&	38.3	&	\multicolumn{1}{|c}{-} & \multicolumn{1}{|c}{-}		\\\cline{3-8}
&&	length (\%)			&	6.1	&	57.8	&	36.1	&	\multicolumn{1}{|c}{-}	&	\multicolumn{1}{|c}{-} \\\cline{3-8}
&&	\# active threads	&	2.7	&	0.1	&	2.5	&	\multicolumn{1}{|c}{-}	&	\multicolumn{1}{|c}{-} \\
%\hline 
%\multirow{3}*{Chess}	 	&
%&	CPU util. (\%)		&	52.8	&	4.4	&	46.1	&	\multicolumn{1}{|c}{-} & \multicolumn{1}{|c}{-}	\\\cline{3-8}
%&&	length (\%)			&	3.8	&	65.3	&	30.9	&	\multicolumn{1}{|c}{-} & \multicolumn{1}{|c}{-}	\\\cline{3-8}
%&&	\# active threads	&			&			&			&			&			\\
\hline 
\multirow{3}*{Dropbox}	 	&	\multirow{3}{100pt}{Browse files in cloud storage}
&	CPU util. (\%)		&	80.0	&	5.0	&	32.2 	&	\multicolumn{1}{|c}{-} & \multicolumn{1}{|c}{-}	\\\cline{3-8}
&&	length (\%)			&	3.3	&	76.6	&	20.1	&	\multicolumn{1}{|c}{-} & \multicolumn{1}{|c}{-}	\\\cline{3-8}
&&	\# active threads	&	1.1	&	0.0	&	1.1	&	\multicolumn{1}{|c}{-} & \multicolumn{1}{|c}{-}	\\
\hline 
\multirow{3}*{Estrongs} 	&	\multirow{3}{100pt}{Search a keyword in local files} 
&	CPU util. (\%)		& 	40.6	&	4.8	&	25.1	&	\multicolumn{1}{|c}{-} & \multicolumn{1}{|c}{-}		\\\cline{3-8}
&&	length (\%)			&	 3.1	&	78.5	&	18.4	&	\multicolumn{1}{|c}{-} & \multicolumn{1}{|c}{-}		\\\cline{3-8}
&&	\# active threads	&	1.0	&	0.0	&	1.9	&	\multicolumn{1}{|c}{-} & \multicolumn{1}{|c}{-}		\\
\hline 
\multirow{3}*{Facebook} 	&	\multirow{3}{100pt}{Read articles in social network service}
&	CPU util. (\%)		&	68.3	&	26.7	&	56.3	&	\multicolumn{1}{|c}{-} & \multicolumn{1}{|c}{-}		\\\cline{3-8}
&&	length (\%)			&	12.2	&	64.6	&	23.2	&	\multicolumn{1}{|c}{-} & \multicolumn{1}{|c}{-}		\\\cline{3-8}
&&	\# active threads	&	2.3	&	1.0	&	1.7	&	\multicolumn{1}{|c}{-} & \multicolumn{1}{|c}{-}		\\
\hline 
\multirow{3}*{Map}	 		&	\multirow{3}{100pt}{Search a place and magnify its neighborhood with Google map}
&	CPU util. (\%)		&	87.3	&	7.8	&	44.7	&	\multicolumn{1}{|c}{-} & \multicolumn{1}{|c}{-}		\\\cline{3-8}
&&	length (\%)			&	11.6	&	49.0	&	39.4	&	\multicolumn{1}{|c}{-} & \multicolumn{1}{|c}{-}		\\\cline{3-8}
&&	\# active threads	&	2.6	&	0.2	&	2.5	&	\multicolumn{1}{|c}{-} & \multicolumn{1}{|c}{-}		\\
\hline 
\multirow{3}*{Virus}	 	&	\multirow{3}{100pt}{Scan viruses in local disk and see the results}
&	CPU util. (\%)		&	53.8	&	5.6	&	55.9	&	\multicolumn{1}{|c}{-} & \multicolumn{1}{|c}{-}		\\\cline{3-8}
&&	length (\%)			&	3.2	&	64.0	&	32.8	&	\multicolumn{1}{|c}{-} & \multicolumn{1}{|c}{-}		\\\cline{3-8}
&&	\# active threads	&	0.8	&	0.1	&	2.0	&	\multicolumn{1}{|c}{-} & \multicolumn{1}{|c}{-}		\\
\hline \hline
\end{tabular}
\end{footnotesize}
\end{center}
%\vspace*{-0.15in}
\caption{Summary of Application Phase Characteristics (two 1GHz cores)}
%\vspace*{-0.2in}
\label{tab:summary}
\end{table*}

Table~\ref{tab:summary} presents the utilization and TLP for our benchmark applications in
different phases. For each phase of an application, three numbers are shown. Firstly,
CPU utilization is an average utilization for two cores during the phase. Secondly, 
length shows the portion of execution time for each phase. The third low, the number of
active threads presents the available parallelism. For the responsive and interactive 
phases, a phase is further decomposed into a portion of the phase with and without 
user interactions. In interactive phases, the touch-based decomposition exhibit little difference.

In the table, the first four applications exhibit interactive phases, when CPUs are active
even without user touch events. 
The rest of applications actively use CPUs only for
setup phases, and for processing user touch events.
Across all the applications, setup phases exhibit high CPU utilization, although the percentage
for the length is relatively small, as the setup phases occur infrequently and their lengths are short.

The available TLP is in general small for most of the applications. 
Some applications have more than one active threads, although the average utilization on both cores are
relatively small under 50-60\%. Active power management must lower the clock frequency to match the
low utilization. However, whether both cores are used with low clock frequency is highly dependent upon
the TLP load balance between the two cores, since the frequency of each core cannot be changed independently.
