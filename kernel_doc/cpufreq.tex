
\section{CPUFreq 분석}


\subsection{Ondemand Governor}


\subsubsection{주요 설정 parameter}


\textbf{powersave\_bias} \\
기본적으로(powersave\_bias가 0인 경우) Ondemand governor는 idle time을 가장 최소로 workload를 마칠 수 있는 minimum frequency를 선택한다. 
하지만 더 많은 power saving을 원하는 경우, powersave\_bias가 설정되어 있으면 기본적으로 Ondemand governor가 설정하는 frequency보다 더 낮게 설정된다. 
이 값은 1 - 1000 사이이며, 이 값에 따라서 frequency가 낮아지게 된다. 
예를 들어, powersave\_bias가 100으로 설정되었고, governor는 2000MHz를 frequency로 선택한 경우, 2000 - (2000 *100/1000) = 1800MHz가 다음에 수행될 frequency로 선택되며, 
HW의 지원여부에 따라서 가장 가까운 frequency로 설정된다. 
%Reference: http://publib.boulder.ibm.com/infocenter/lnxinfo/v3r0m0/index.jsp?topic=%2Fliaai%2Fcpufreq%2FTheOndemandGovernor.htm

\begin{comment}
\begin{compactenum}
  \item powersavei\_bias
    \begin{compactenum}
      \item 
    \end{compactenum}
\end{compactenum}
\end{comment}



\subsubsection{주요 구조체 및 상수}

\begin{lstlisting}
#define DEF_FREQUENCY_DOWN_DIFFERENTIAL     (10)
#define DEF_FREQUENCY_UP_THRESHOLD      (80)
#define DEF_SAMPLING_DOWN_FACTOR        (1)
#define MAX_SAMPLING_DOWN_FACTOR        (100000)
#define MICRO_FREQUENCY_DOWN_DIFFERENTIAL   (3)
#define MICRO_FREQUENCY_UP_THRESHOLD        (95)
#define MICRO_FREQUENCY_MIN_SAMPLE_RATE     (10000)
#define MIN_FREQUENCY_UP_THRESHOLD      (11)
#define MAX_FREQUENCY_UP_THRESHOLD      (100)

#define LATENCY_MULTIPLIER          (1000)
#define MIN_LATENCY_MULTIPLIER          (100)
#define TRANSITION_LATENCY_LIMIT        (10 * 1000 * 1000)
\end{lstlisting}

\vspace{\baselineskip}

구조체 dbs\_tuners의 인스턴스 dbs\_tuners\_ins는 전역으로 1개만 존재하며, tickless kernel인 경우에는 up\_threshold와 같은 설정 값이 매우 tight하게 설정된다. 
\begin{lstlisting}
static struct dbs_tuners {
    unsigned int sampling_rate;
    unsigned int up_threshold;
    unsigned int down_differential;
    unsigned int ignore_nice;
    unsigned int sampling_down_factor;
    unsigned int powersave_bias;
    unsigned int io_is_busy;
    struct notifier_block dvfs_lat_qos_db;
    unsigned int dvfs_lat_qos_wants;
    unsigned int freq_step;
#ifdef CONFIG_CPU_FREQ_GOV_ONDEMAND_FLEXRATE
    unsigned int flex_sampling_rate;
    unsigned int flex_duration;
#endif
} dbs_tuners_ins = {
    .up_threshold = DEF_FREQUENCY_UP_THRESHOLD,   /* 80 */
    .sampling_down_factor = DEF_SAMPLING_DOWN_FACTOR, /* 1 */
    .down_differential = DEF_FREQUENCY_DOWN_DIFFERENTIAL,
    .ignore_nice = 0, 
    .powersave_bias = 0, 
    .freq_step = 100, 
};
\end{lstlisting}

\vspace{\baselineskip}

또한 callback 함수 호출시 인자로 자주 전달되는 cpufreq\_policy 구조체는 다음과 같다.
SMP 더라도 regulator의 문제로 인해서 각 core간에 frequency가 공유되는 경우가 많다.
Per-Core DVFS가 되는 CPU는 AMD와 Qualcomm의 Snapdragon만 현재 가능하며, 이 경우에도  Per-Core로 regulator가 존재하는 것이 아니기 때문에, 가장 높은 frequency에 dominant하게 power consumption이 결정된다. 
\begin{lstlisting}
struct cpufreq_policy { 
    cpumask_var_t       cpus;   /* CPUs requiring sw coordination */
    cpumask_var_t       related_cpus; /* CPUs with any coordination */
    unsigned int        shared_type; /* ANY or ALL affected CPUs
                        should set cpufreq */
    unsigned int        cpu;    /* cpu nr of registered CPU */
    struct cpufreq_cpuinfo  cpuinfo;/* see above */

    unsigned int        min;    /* in kHz */
    unsigned int        max;    /* in kHz */
    unsigned int        cur;    /* in kHz, only needed if cpufreq
                     * governors are used */
    unsigned int        policy; /* see above */
    struct cpufreq_governor *governor; /* see below */

    struct work_struct  update; /* if update_policy() needs to be
                     * called, but you're in IRQ context */

    struct cpufreq_real_policy  user_policy;

    struct kobject      kobj;
    struct completion   kobj_unregister;
};
\end{lstlisting}

\vspace{\baselineskip}

각 core 별로는 Per core data로 다음 구조체를 각각 가지고 있다. 
\begin{lstlisting}
struct cpu_dbs_info_s {
    cputime64_t prev_cpu_idle;
    cputime64_t prev_cpu_iowait;
    cputime64_t prev_cpu_wall;
    cputime64_t prev_cpu_nice;
    struct cpufreq_policy *cur_policy;      
    struct delayed_work work;       /* delayed worker */
    struct cpufreq_frequency_table *freq_table;           /* CPU frequency table */
    unsigned int freq_lo;
    unsigned int freq_lo_jiffies;
    unsigned int freq_hi_jiffies;
    unsigned int rate_mult;
    int cpu;
    unsigned int sample_type:1;   /* DBS_NORMAL_SAMPLE, DBS_SUB_SAMPLE */
    /*
     * percpu mutex that serializes governor limit change with
     * do_dbs_timer invocation. We do not want do_dbs_timer to run
     * when user is changing the governor or limits.
     */
    struct mutex timer_mutex;   /* mutex lock */
    bool activated;       /* dbs_timer_init is in effect */
#ifdef CONFIG_CPU_FREQ_GOV_ONDEMAND_FLEXRATE
    unsigned int flex_duration;
#endif
};
static DEFINE_PER_CPU(struct cpu_dbs_info_s, od_cpu_dbs_info);
\end{lstlisting}

\vspace{\baselineskip}

그외 sysfs에 등록되는 데이터는 다음과 같이 선언되어 있다. 
\begin{lstlisting}
cpufreq_freq_attr_ro_perm(cpuinfo_cur_freq, 0400);
cpufreq_freq_attr_ro(cpuinfo_min_freq);
cpufreq_freq_attr_ro(cpuinfo_max_freq);
cpufreq_freq_attr_ro(cpuinfo_transition_latency);
cpufreq_freq_attr_ro(scaling_available_governors);
cpufreq_freq_attr_ro(scaling_driver);
cpufreq_freq_attr_ro(scaling_cur_freq);
cpufreq_freq_attr_ro(bios_limit);
cpufreq_freq_attr_ro(related_cpus);
cpufreq_freq_attr_ro(affected_cpus);
cpufreq_freq_attr_rw(scaling_min_freq);
cpufreq_freq_attr_rw(scaling_max_freq);
cpufreq_freq_attr_rw(scaling_governor); 
cpufreq_freq_attr_rw(scaling_setspeed);

static struct attribute *default_attrs[] = {
    &cpuinfo_min_freq.attr,
    &cpuinfo_max_freq.attr,
    &cpuinfo_transition_latency.attr,
    &scaling_min_freq.attr,
    &scaling_max_freq.attr,
    &affected_cpus.attr,
    &related_cpus.attr,
    &scaling_governor.attr,
    &scaling_driver.attr,
    &scaling_available_governors.attr,
    &scaling_setspeed.attr,
    NULL
};

struct kobject *cpufreq_global_kobject;
EXPORT_SYMBOL(cpufreq_global_kobject);
\end{lstlisting}

\vspace{\baselineskip}

그외 직접적으로는 관련있는 것은 아니지만, 참고할만한 구조체로 cpufreq\_driver가 있다. 
이 구조체는 CPU vendor에서 작업하는 것으로 Exynos의 경우, arch/arm/mach-exynos에 위치하고 있다. 
CPU frequency를 설정하거나, 읽거나, 확인하는 것과 같은 low level의 작업을 처리한다. 
\begin{lstlisting}
struct cpufreq_driver {
    struct module           *owner;
    char            name[CPUFREQ_NAME_LEN];
    u8          flags;
                     
    /* needed by all drivers */
    int (*init)     (struct cpufreq_policy *policy);
    int (*verify)   (struct cpufreq_policy *policy);
    
    /* define one out of two */
    int (*setpolicy)    (struct cpufreq_policy *policy);
    int (*target)   (struct cpufreq_policy *policy,     /* Set CPU frequency*/
                 unsigned int target_freq,
                 unsigned int relation);

    /* should be defined, if possible */
    unsigned int    (*get)  (unsigned int cpu);         /* Return CPU frequency */

    /* optional */
    unsigned int (*getavg)  (struct cpufreq_policy *policy,
                 unsigned int cpu);
    int (*bios_limit)   (int cpu, unsigned int *limit);

    int (*exit)     (struct cpufreq_policy *policy);
    int (*suspend)  (struct cpufreq_policy *policy);
    int (*resume)   (struct cpufreq_policy *policy);
    struct freq_attr    **attr;
};  
\end{lstlisting}



\subsubsection{초기화 작업}
Ondemand driver의 init 함수에서는 앞서 설명한 것처럼 tickless kernel인 경우, 설정 값이 매우 tight하게 설정한다. 
즉 dbs\_tuners\_ins의 up\_threshold는 95로, down\_differential는 3으로 설정한다.
이 값은 추후 delayed work에 의해서 frequency를 변경할 때에 사용된다. 
그 이후 QoS의 설정이 변경시 notification을 받기 위한 callback 함수 등록과 
Ondemand governor를 등록한다. 

\begin{lstlisting}
static int __init cpufreq_gov_dbs_init(void)
{
    cputime64_t wall;
    u64 idle_time;
    int cpu = get_cpu();
    int err = 0;

    /* In case of CONFIG_NO_HZ is not defined, get_cpu_idle_time_us() return -1 */
    idle_time = get_cpu_idle_time_us(cpu, &wall);
    put_cpu();
    if (idle_time != -1ULL) {
        /* In case of CONFIG_NO_HZ is defined, in other words, tickless kernel */
        /* Idle micro accounting is supported. Use finer thresholds */
        dbs_tuners_ins.up_threshold = MICRO_FREQUENCY_UP_THRESHOLD;   /* 95 */
        dbs_tuners_ins.down_differential =
                    MICRO_FREQUENCY_DOWN_DIFFERENTIAL;                /* 3 */
        /*
         * In no_hz/micro accounting case we set the minimum frequency
         * not depending on HZ, but fixed (very low). The deferred
         * timer might skip some samples if idle/sleeping as needed.
        */
        min_sampling_rate = MICRO_FREQUENCY_MIN_SAMPLE_RATE;          /* 10000 */
    } else {
        /* For correct statistics, we need 10 ticks for each measure */
        /* In case of non tickless kernel */
        min_sampling_rate =
            MIN_SAMPLING_RATE_RATIO * jiffies_to_usecs(10);
    }

    err = pm_qos_add_notifier(PM_QOS_DVFS_RESPONSE_LATENCY,
                &ondemand_qos_dvfs_lat_nb);
    if (err)
        return err;

    /* Register CPU governor */
    err = cpufreq_register_governor(&cpufreq_gov_ondemand);
    if (err) {
        pm_qos_remove_notifier(PM_QOS_DVFS_RESPONSE_LATENCY,
                       &ondemand_qos_dvfs_lat_nb);
    }

    return err;
}
\end{lstlisting}

\vspace{\baselineskip}

cpufreq\_register\_governor() 함수는 cpufreq\_governor 구조체를 인자로 받으며, Ondemand의 경우 다음과 같이 해당 구조체가 설정되었다. 
여기서 중요한 인자는 cpufreq\_governor\_dbs로 함수 포인터로 governor가 설정/해지/변경 될때마다 해당 함수가 callback으로 호출된다. 
\begin{lstlisting}
#ifndef CONFIG_CPU_FREQ_DEFAULT_GOV_ONDEMAND
static
#endif
struct cpufreq_governor cpufreq_gov_ondemand = {
       .name                   = "ondemand",
       .governor               = cpufreq_governor_dbs,
       .max_transition_latency = TRANSITION_LATENCY_LIMIT,  
                              /* 10*1000*1000 */
       .owner                  = THIS_MODULE,
};
\end{lstlisting}


\subsubsection{Governor callback 함수}

cpufreq\_governor\_dbs() 함수가 호출되는 시점은 1) 특정 core에서 해당 governor가 처음 수행될 때(CPUFREQ\_GOV\_START), 특정 core에서 해당 governor가 종료될 때(CPUFREQ\_GOV\_STOP), 특정 core의 min, max frequency가 변경(CPUFREQ\_GOV\_LIMITS)될 때이다. 
이 값은 integer 정수로 전달되기 때문에, 대부분의 경우 switch - case 구문으로 처리한다. 
초기화 단계에서 cpufreq\_governor\_dbs() 함수의 동작은 다음과 같다. 
\vspace{\baselineskip}  % http://diegotorquemada-latex.blogspot.kr/2007/02/and-vspacebaselineskip.html
\begin{compactenum}
\item 현재 수행 중인 CPU와 per core data를 각각 cpu, this\_dbs\_info에 저장한다. 이 과정은 다른 모든 경우에도 동일하게 수행된다.
\item 전역으로 mutex lock을 걸고, frequency를 공유하는 모든 core의 per core data를 입력한다. 
입력하는 데이터는 policy, 각 core의 load를 측정하기 위한 wall time, idle time 값이다. 
\item 해당 governor를 사용하는 첫 core인 경우(dbs\_enable이 1인 경우), sysfs에 필요한 데이터를 등록하고, 
dbs\_tuners\_ins의 sampling\_rate을 계산한다. 
\item mutex lock을 풀고, 각 core에서 사용하는 mutex를 초기화 하고, 각 core별로 수행하는 timer를 초기화 한다.
\end{compactenum}

\begin{lstlisting}
static int cpufreq_governor_dbs(struct cpufreq_policy *policy,
                   unsigned int event)
{

    unsigned int cpu = policy->cpu;
    struct cpu_dbs_info_s *this_dbs_info;
    unsigned int j;
    int rc;

    this_dbs_info = &per_cpu(od_cpu_dbs_info, cpu);

    /* cpu: current core
       this_dbs_info: per core data of current core */
    switch (event) {
    case CPUFREQ_GOV_START:
        /* abnormal case */
        if ((!cpu_online(cpu)) || (!policy->cur))
            return -EINVAL;

        mutex_lock(&dbs_mutex);

        dbs_enable++;

        /* for all cores which shared frequency, 
        Set idle & wall time in per cpu data for calculating core load */
        for_each_cpu(j, policy->cpus) {
            struct cpu_dbs_info_s *j_dbs_info;
            j_dbs_info = &per_cpu(od_cpu_dbs_info, j);

            /* Set policy */
            j_dbs_info->cur_policy = policy;

            /* Set idle time & wall time of each core */
            j_dbs_info->prev_cpu_idle = get_cpu_idle_time(j,
                        &j_dbs_info->prev_cpu_wall);
      
            /* in case of ignore_nice is set, 
            then save the previous nice value */
            /* TODO Check kstat module */
            if (dbs_tuners_ins.ignore_nice) {
                j_dbs_info->prev_cpu_nice =
                        kstat_cpu(j).cpustat.nice;
            }
        }
        this_dbs_info->cpu = cpu;
        this_dbs_info->rate_mult = 1;

        /* Get freq_table of each core & Set freq_lo as 0 */
        ondemand_powersave_bias_init_cpu(cpu);


        /*
         * Start the timerschedule work, when this governor
         * is used for first time
         */
        if (dbs_enable == 1) {
            unsigned int latency;

            rc = sysfs_create_group(cpufreq_global_kobject,
                        &dbs_attr_group);
            if (rc) {
                mutex_unlock(&dbs_mutex);
                return rc;
            }

            /* Compute dbs_tuners_ins.sampling_rate */
            /* policy latency is in nS. Convert it to uS first */
            latency = policy->cpuinfo.transition_latency / 1000;
            if (latency == 0)
                latency = 1;
            /* Bring kernel and HW constraints together */
            min_sampling_rate = max(min_sampling_rate,
                    MIN_LATENCY_MULTIPLIER * latency);
            dbs_tuners_ins.sampling_rate =
                max(min_sampling_rate,
                    latency * LATENCY_MULTIPLIER);

            /* Only x86: In case of ARM, io_is_busy is 0 */
            dbs_tuners_ins.io_is_busy = should_io_be_busy();
        }
        mutex_unlock(&dbs_mutex);
        mutex_init(&this_dbs_info->timer_mutex);

        /* Set per core timer */
        dbs_timer_init(this_dbs_info);
        break;

    case CPUFREQ_GOV_STOP:
        ...
        break;

    case CPUFREQ_GOV_LIMITS:
        ...
        break;


\end{lstlisting}


\subsubsection{Timer 관련 함수}
timer를 초기화하는 dbsi\_timer\_init() 함수에서는 적절한 sampling rate을 계산하여,
delayed worker를 이용하여 do\_dbs\_timer()가 delay jiffies 이후에 호출되도록 한다. 
\begin{lstlisting}
static inline void dbs_timer_init(struct cpu_dbs_info_s *dbs_info)
{
    /* find approprite sampling rate(usec) and convert into jiffies */
    /* We want all CPUs to do sampling nearly on same jiffy */
    int delay = usecs_to_jiffies(effective_sampling_rate());

    if (num_online_cpus() > 1)
        delay -= jiffies % delay;

    /* Add delayed worker
    do_dbs_timer() function is called after delay jiffies */
    dbs_info->sample_type = DBS_NORMAL_SAMPLE;
    INIT_DELAYED_WORK_DEFERRABLE(&dbs_info->work, do_dbs_timer);
    schedule_delayed_work_on(dbs_info->cpu, &dbs_info->work, delay);
    dbs_info->activated = true;
}
\end{lstlisting}

버전마다 조금씩 다르지만 sampling rate, 즉 delay할 시간을 결정할 때에
effective\_sampling\_rate() 함수와 같이 QoS 등을 고려하여 가장 적절한 sampling rate을 찾아서 반환하는 경우도 있다. 
\begin{lstlisting}
static unsigned int effective_sampling_rate(void)
{
    unsigned int effective;

    if (dbs_tuners_ins.dvfs_lat_qos_wants)
        effective = min(dbs_tuners_ins.dvfs_lat_qos_wants,
                dbs_tuners_ins.sampling_rate);
    else
        effective = dbs_tuners_ins.sampling_rate;

    return max(effective, min_sampling_rate);
}
\end{lstlisting}
do\_dbs\_timer() 함수는 sampling rate 마다 주기적으로 호출되는 함수이다. 
이 함수에서는 policy를 공유하는 core의 load에 맞추어 frequency를 조절하고, 
다시 delayed worker에 재등록하는 동작을 반복한다. 세부 동작은 다음과 같다. 

\begin{compactenum}
\item 코드가 수행 중인 cpu, per cpu data, sample\_type을 가지고 온다. 
\item powersave\_bias가 설정되어 있지 않거나, sample\_type이 DBS\_NORMAL\_SAMPLE인지 확인한다. 
대부분의 경우가 이에 해당한다. 
\item policy를 공유하는 모든 core의 load를 고려하여 dbs\_check\_cpu() 함수에서 frequency를 변경한다. 
이 함수는 다음에 좀 더 자세히 설명한다. 
\item dbs\_info->freqi\_lo가 설정되어 있는 경우에는, timer의 delay를 freq\_hi\_jiffies으로 설정한다. 
즉 해당 core의 sampling rate을 변경하는 것이다. powersave\_bias\_target() 함수에서 dbs\_info->freq\_lo를 설정한다.
\item dbs\_info->freq\_lo가 설정되어 있지 않은 경우, 즉 대부분의 경우에는 동일한 sampling rate으로 timer를 설정한다. 
이때 jiffies 단위로 지연시킬 시간을 정렬한다. 
\item powersave\_bias가 설정되었거나, DBS\_SUB\_SAMPLE인 경우, frequency는 무조건 dbs\_info->freq\_lo로 설정된다.
\item 다음에 수행할 delayed worker에 재등록한다. 
\end{compactenum}

\begin{lstlisting}
static void do_dbs_timer(struct work_struct *work)
{
    /* 1) Get cpu, cpu_dbs_info_s data of cpu & sample_type */
    struct cpu_dbs_info_s *dbs_info =
        container_of(work, struct cpu_dbs_info_s, work.work);
    unsigned int cpu = dbs_info->cpu;
    int sample_type = dbs_info->sample_type;

    int delay;

    mutex_lock(&dbs_info->timer_mutex);
    dbs_info->sample_type = DBS_NORMAL_SAMPLE;

    /* 2) Check both powersave_bias and sample_type */
    if (!dbs_tuners_ins.powersave_bias ||
        sample_type == DBS_NORMAL_SAMPLE) {

	      /* 3) Set the cpu frequency according to the max load of cpus */
        dbs_check_cpu(dbs_info);

        /* 4) When freq_lo is set, each core has its own sampling rate */
        if (dbs_info->freq_lo) {
            /* Setup timer for SUB_SAMPLE */
            dbs_info->sample_type = DBS_SUB_SAMPLE;
            delay = dbs_info->freq_hi_jiffies;
        } else {
            /* We want all CPUs to do sampling nearly on
             * same jiffy
             */
            /* 5) every core has the same sampling rate */
            delay = usecs_to_jiffies(effective_sampling_rate()
                * dbs_info->rate_mult);

            if (num_online_cpus() > 1)
                delay -= jiffies % delay;
        }
    } else {
        /* 6) When powersave_bias is set, then frequency always set freq_lo */
        __cpufreq_driver_target(dbs_info->cur_policy,
            dbs_info->freq_lo, CPUFREQ_RELATION_H);
        delay = dbs_info->freq_lo_jiffies;
    }


    /* 7) Reregister the timer function again */
    schedule_delayed_work_on(cpu, &dbs_info->work, delay);
    mutex_unlock(&dbs_info->timer_mutex);
}
\end{lstlisting}

\subsubsection{cpu의 load 계산 및 frequency 변경}
dbs\_check\_cpu() 함수에서 policy를 공유하는 core들을 확인하고, 실제 frequency를 변경하는 작업을 한다. 
이 함수의 세부 동작은 다음과 같다. 
\begin{compactenum}
\item 현재 core의 cur\_wall\_time, cur\_idle\_time, cur\_iowait\_time를 측정하여
sampling rate 동안의 delta인 wall\_time, idle\_time, iowait\_time을 계산한다. 
측정한 값은 다음 턴을 위해서 per cpu data에 저장한다.
\item 현재 core의 frequency를 고려한 load를 계산한다. (0 $\sim$ 100)
\item sampling rate 동안의 평균 frequency를 cpu driver를 이용하여 측정한다
Exynos와 같이 지원하지 않는 경우, 현재 freqnecy를 대신 사용한다.
\item frequency와 무관한 core 자체에 걸린 load를 계산(load\_freq)를 계산하고, 
policy를 공유하는 core 중에서 최대값을 max\_load\_freq에 저장한다. 
예를 들어 1000MHz의 50\%인 경우와 500MHz의 100\%는 동일한 load\_freq를 갖는다.
\item max\_load\_freq가 현재 frequency의 up\_threshold를 넘었는지 확인한다. 
tickless kernel인 경우에는 up\_threshold가 95이기 때문에, 
max\_load\_freq가 현재 frequency에서 95\%이상의 load가 되는지 확인하는 것이다. 
만약 그런 경우, 변경할 frequency(target)을 max로 변경하고 종료한다. 
\item 특별히 Exynos에서는 현재 frequency가 min이고, frequency를 낮춰야하는 경우, 
더 이상 frequency를 낮출수 없기 때문에 종료한다. 
\item max\_load\_freq가 현재 frequency의 (up\_threshold-down\_differential)보다 작은지 확인한다. 
만약 그런 경우, 변경할 freqneucy(freq\_next)를 계산하여 설정한다. 
예를 들어, 현재 1000Mhz로 동작 중인 CPU의 load가 30\%이면, max\_load\_freq = 1000 * 30 = 30000이 된다. 
그렇기 때문에 freq\_next는 30000/92 = 326이 되고, CPUFREQ\_RELATION\_L로 인하여 300MHz로 설정이 된다. 
powersave\_bias가 설정된 경우에는 powersave\_bias\_target() 함수에 의해서 다음에 설정할 frequency가 결정된다.
\end{compactenum}

\begin{lstlisting}
static void dbs_check_cpu(struct cpu_dbs_info_s *this_dbs_info)
{
    unsigned int max_load_freq;

    struct cpufreq_policy *policy;
    unsigned int j;

    this_dbs_info->freq_lo = 0;
    policy = this_dbs_info->cur_policy;

    /*
     * Every sampling_rate, we check, if current idle time is less
     * than 20% (default), then we try to increase frequency.
     * Every sampling_rate, we look for a the lowest
     * frequency which can sustain the load while keeping idle time over
     * 30%. If such a frequency exist, we try to decrease to this frequency.
     *
     * Any frequency increase takes it to the maximum frequency.
     * Frequency reduction happens at minimum steps of
     * 5% (default) of current frequency
     */

    /* Get Absolute Load - in terms of freq */
    max_load_freq = 0;

    /* iterate each cpu that is shared the policy */
    for_each_cpu(j, policy->cpus) {

        /* 1) calculate wall_time, idle_time, iowait_time */
        struct cpu_dbs_info_s *j_dbs_info;
        cputime64_t cur_wall_time, cur_idle_time, cur_iowait_time;
        unsigned int idle_time, wall_time, iowait_time;
        unsigned int load, load_freq;
        int freq_avg;

        j_dbs_info = &per_cpu(od_cpu_dbs_info, j);

        cur_idle_time = get_cpu_idle_time(j, &cur_wall_time);
        cur_iowait_time = get_cpu_iowait_time(j, &cur_wall_time);

        wall_time = (unsigned int) cputime64_sub(cur_wall_time,
                j_dbs_info->prev_cpu_wall);
        j_dbs_info->prev_cpu_wall = cur_wall_time;

        idle_time = (unsigned int) cputime64_sub(cur_idle_time,
                j_dbs_info->prev_cpu_idle);
        j_dbs_info->prev_cpu_idle = cur_idle_time;

        iowait_time = (unsigned int) cputime64_sub(cur_iowait_time,
                j_dbs_info->prev_cpu_iowait);
        j_dbs_info->prev_cpu_iowait = cur_iowait_time;

        if (dbs_tuners_ins.ignore_nice) {
            cputime64_t cur_nice;
            unsigned long cur_nice_jiffies;

            cur_nice = cputime64_sub(kstat_cpu(j).cpustat.nice,
                     j_dbs_info->prev_cpu_nice);
            /*
             * Assumption: nice time between sampling periods will
             * be less than 2^32 jiffies for 32 bit sys
             */
            cur_nice_jiffies = (unsigned long)
                    cputime64_to_jiffies64(cur_nice);

            j_dbs_info->prev_cpu_nice = kstat_cpu(j).cpustat.nice;
            idle_time += jiffies_to_usecs(cur_nice_jiffies);
        }

        /*
         * For the purpose of ondemand, waiting for disk IO is an
         * indication that you're performance critical, and not that
         * the system is actually idle. So subtract the iowait time
         * from the cpu idle time.
         */
	
	/* io_is_busy is for only x86. Not support for ARM */
        if (dbs_tuners_ins.io_is_busy && idle_time >= iowait_time)
            idle_time -= iowait_time;

        if (unlikely(!wall_time || wall_time < idle_time))
            continue;

	/* 2) calculate the load considering core's frequency */
        load = 100 * (wall_time - idle_time) / wall_time;

        /* 3) get average frequency during sampling rate */
        freq_avg = __cpufreq_driver_getavg(policy, j);
        if (freq_avg <= 0)
            freq_avg = policy->cur;

        /* 4) calculate absolute load and 
        find max_load_freq among absolute load of all shared policy core */
        load_freq = load * freq_avg;
        if (load_freq > max_load_freq)
            max_load_freq = load_freq;
    }

    /* 5) Check the frequency should be raised */
    if (max_load_freq > dbs_tuners_ins.up_threshold * policy->cur) {
	
        int inc = (policy->max * dbs_tuners_ins.freq_step) / 100;
        int target = min(policy->max, policy->cur + inc);

        if (policy->cur < policy->max && target == policy->max)
            this_dbs_info->rate_mult =
                dbs_tuners_ins.sampling_down_factor;

        dbs_freq_increase(policy, target);
        return;
    }

    /* 6) Exynos turning */
#if !defined(CONFIG_ARCH_EXYNOS4) && !defined(CONFIG_ARCH_EXYNOS5)
    /* if we cannot reduce the frequency anymore, break out early */
    if (policy->cur == policy->min)
        return;
#endif

    /*
     * The optimal frequency is the frequency that is the lowest that
     * can support the current CPU usage without triggering the up
     * policy. To be safe, we focus 10 points under the threshold.
     */

    /* 7) Check the frequency should be lowed */
    if (max_load_freq <
        (dbs_tuners_ins.up_threshold - dbs_tuners_ins.down_differential) *
         policy->cur) {
        unsigned int freq_next;

        freq_next = max_load_freq /
                (dbs_tuners_ins.up_threshold -
                 dbs_tuners_ins.down_differential);

        /* No longer fully busy, reset rate_mult */
        this_dbs_info->rate_mult = 1;

        if (freq_next < policy->min)
            freq_next = policy->min;

        if (!dbs_tuners_ins.powersave_bias) {
            __cpufreq_driver_target(policy, freq_next,
                    CPUFREQ_RELATION_L);
        } else {
            int freq = powersave_bias_target(policy, freq_next,
                    CPUFREQ_RELATION_L);
            __cpufreq_driver_target(policy, freq,
                CPUFREQ_RELATION_L);
        }
    }
}
\end{lstlisting}

powersave\_bias가 설정된 경우에는 powersave\_bias\_target() 함수에 의해서 다음에 설정할 frequency가 결정된다.
\begin{lstlisting}
/*
 * Find right freq to be set now with powersave_bias on.
 * Returns the freq_hi to be used right now and will set freq_hi_jiffies,
 * freq_lo, and freq_lo_jiffies in percpu area for averaging freqs.
 */
static unsigned int powersave_bias_target(struct cpufreq_policy *policy,
                      unsigned int freq_next,
                      unsigned int relation)
{
    unsigned int freq_req, freq_reduc, freq_avg;
    unsigned int freq_hi, freq_lo;
    unsigned int index = 0;
    unsigned int jiffies_total, jiffies_hi, jiffies_lo;
    struct cpu_dbs_info_s *dbs_info = &per_cpu(od_cpu_dbs_info,
                           policy->cpu);

    if (!dbs_info->freq_table) {
        dbs_info->freq_lo = 0;
        dbs_info->freq_lo_jiffies = 0;
        return freq_next;
    }

    cpufreq_frequency_table_target(policy, dbs_info->freq_table, freq_next,
            relation, &index);
    freq_req = dbs_info->freq_table[index].frequency;
    freq_reduc = freq_req * dbs_tuners_ins.powersave_bias / 1000;
    freq_avg = freq_req - freq_reduc;

    /* Find freq bounds for freq_avg in freq_table */
    index = 0;
    cpufreq_frequency_table_target(policy, dbs_info->freq_table, freq_avg,
            CPUFREQ_RELATION_H, &index);
    freq_lo = dbs_info->freq_table[index].frequency;
    index = 0;
    cpufreq_frequency_table_target(policy, dbs_info->freq_table, freq_avg,
            CPUFREQ_RELATION_L, &index);
    freq_hi = dbs_info->freq_table[index].frequency;

    /* Find out how long we have to be in hi and lo freqs */
    if (freq_hi == freq_lo) {
        dbs_info->freq_lo = 0;
        dbs_info->freq_lo_jiffies = 0;
        return freq_lo;
    }
    jiffies_total = usecs_to_jiffies(effective_sampling_rate());
    jiffies_hi = (freq_avg - freq_lo) * jiffies_total;
    jiffies_hi += ((freq_hi - freq_lo) / 2);
    jiffies_hi /= (freq_hi - freq_lo);
    jiffies_lo = jiffies_total - jiffies_hi;
    dbs_info->freq_lo = freq_lo;
    dbs_info->freq_lo_jiffies = jiffies_lo;
    dbs_info->freq_hi_jiffies = jiffies_hi;
    return freq_hi;
}
\end{lstlisting}


\subsubsection{CPUFREQ\_GOV\_STOP 이벤트 처리}
이제 cpufreq\_governor\_dbs() 함수에서 다루지 않은 CPUFREQ\_GOV\_STOP에 대해서 알아보겠다. 
이 이벤트는 더이상 해당 governor가 사용되지 않을 때 발생하는 이벤트이기 때문에, 주기적으로 수행하던 timer 함수 등을 정리해야 한다. 
세부 동작은 다음과 같다. 
\begin{compactenum}
\item 현재 수행 중인 delayed worker를 취소시킨다. 
\item governor 전반에서 쓰이는 변수를 수정할 예정이기 때문에, mutex lock을 설정하고 per cpu data의 timer mutex를 해제하고, dbs\_enable을 하나 감소시킨다. 
\item dbs\_enable가 0인 경우, 즉 더 이상 이 governor를 사용하는 core가 없는 경우, sysfs에서 해당 파일을 제거한다.
\end{compactenum}

\begin{lstlisting}
    case CPUFREQ_GOV_STOP:
        dbs_timer_exit(this_dbs_info);

        mutex_lock(&dbs_mutex);
        mutex_destroy(&this_dbs_info->timer_mutex);
        dbs_enable--;
        mutex_unlock(&dbs_mutex);
        if (!dbs_enable)
            sysfs_remove_group(cpufreq_global_kobject,
                       &dbs_attr_group);

        break;
\end{lstlisting}


\subsubsection{CPUFREQ\_GOV\_LIMITS 이벤트 처리}
마지막으로 governor의 max/min frequency가 변경되었을 때 발생하는 CPUFREQ\_GOV\_LIMITS에 대해서 알아보겠다. 
각 core별로 변경된 policy의 max frequency가 현재 frequency보다 작은 경우, 
변경된 policy의 max frequency로 변경하고, 변경된 policy의 min frequency가 현재 frequency보다  경우, 
변경된 policy의 min frequency로 변경하면 된다. 
\begin{lstlisting}
    case CPUFREQ_GOV_LIMITS:
        mutex_lock(&this_dbs_info->timer_mutex);
        if (policy->max < this_dbs_info->cur_policy->cur)
            __cpufreq_driver_target(this_dbs_info->cur_policy,
                policy->max, CPUFREQ_RELATION_H);
        else if (policy->min > this_dbs_info->cur_policy->cur)
            __cpufreq_driver_target(this_dbs_info->cur_policy,
                policy->min, CPUFREQ_RELATION_L);
        mutex_unlock(&this_dbs_info->timer_mutex);
        break;
    }
\end{lstlisting}
